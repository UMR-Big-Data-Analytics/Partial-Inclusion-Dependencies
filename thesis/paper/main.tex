% VLDB template version of 2020-08-03 enhances the ACM template, version 1.7.0:
% https://www.acm.org/publications/proceedings-template
% The ACM Latex guide provides further information about the ACM template

\documentclass[sigconf, nonacm]{acmart}

%% The following content must be adapted for the final version
% paper-specific
\newcommand\vldbdoi{XX.XX/XXX.XX}
\newcommand\vldbpages{XXX-XXX}
% issue-specific
\newcommand\vldbvolume{14}
\newcommand\vldbissue{1}
\newcommand\vldbyear{2020}
% should be fine as it is
\newcommand\vldbauthors{\authors}
\newcommand\vldbtitle{\shorttitle} 
% leave empty if no availability url should be set
\newcommand\vldbavailabilityurl{https://github.com/Jakob-L-M/partial-inclusion-dependencies}
% whether page numbers should be shown or not, use 'plain' for review versions, 'empty' for camera ready
\newcommand\vldbpagestyle{plain} 

% added packages
\usepackage[noend,linesnumbered]{algorithm2e}
\usepackage{cleveref}

\SetKwInput{KwInput}{Input}
\SetKwInput{KwOutput}{Output}
\SetKwIF{If}{ElseIf}{Else}{if}{}{else if}{else}{end if}

\begin{document}
\title{Partial Inclusion Dependency Discovery}

%%
%% The "author" command and its associated commands are used to define the authors and their affiliations.
\author{Jakob Leander Müller}
\affiliation{%
  \institution{Philipps-Universität Marburg}
  \streetaddress{P.O. Box 1212}
  \city{Marburg}
  \state{Germany}
  \postcode{35037}
}
\email{muelle5t@students.uni-marburg.de}
\email{me@jakob-l-m.de}

\author{Thorsten Papenbrock}
\orcid{0000-0002-4019-8221}
\affiliation{%
  \institution{Philipps-Universität Marburg}
  \city{Marburg}
  \country{Germany}
}
\email{papenbrock@informatik.uni-marburg.de}


%%
%% The abstract is a short summary of the work to be presented in the
%% article.
\begin{abstract}
In a world where data grows exponentially and data sharing becomes more common, the need for efficient and flexible data profiling increases constantly.  Identifying inclusion dependencies (INDs) among attributes is crucial for tasks like data integration, query optimization, and integrity checking. A significant challenge lies in detecting these relationships, particularly with the increasing size of datasets, which escalates the complexity of IND discovery.
Past research dealing with INDs mostly overlooked the potential of partial INDs (pINDs), meaning imperfect subsets. Such pINDs can enable researchers and data engineers to understand databases, even if different sources do not match perfectly. After introducing pIND properties and discussing new strategies for (p)IND discovery, combining known strategies such as a sort-merge join and heavy parallelization, we will introduce \textit{SPIND}. An algorithm that is not only able to find all pINDs, it also outperforms the current state-of-the-art algorithm \textit{BINDER} in both unary and nary settings by multiple magnitudes.
\end{abstract}

\maketitle

%%% do not modify the following VLDB block %%
%%% VLDB block start %%%
\pagestyle{\vldbpagestyle}
\begingroup\small\noindent\raggedright\textbf{PVLDB Reference Format:}\\
\vldbauthors. \vldbtitle. PVLDB, \vldbvolume(\vldbissue): \vldbpages, \vldbyear.\\
\href{https://doi.org/\vldbdoi}{doi:\vldbdoi}
\endgroup
\begingroup
\renewcommand\thefootnote{}\footnote{\noindent
This work is licensed under the Creative Commons BY-NC-ND 4.0 International License. Visit \url{https://creativecommons.org/licenses/by-nc-nd/4.0/} to view a copy of this license. For any use beyond those covered by this license, obtain permission by emailing \href{mailto:info@vldb.org}{info@vldb.org}. Copyright is held by the owner/author(s). Publication rights licensed to the VLDB Endowment. \\
\raggedright Proceedings of the VLDB Endowment, Vol. \vldbvolume, No. \vldbissue\ %
ISSN 2150-8097. \\
\href{https://doi.org/\vldbdoi}{doi:\vldbdoi} \\
}\addtocounter{footnote}{-1}\endgroup
%%% VLDB block end %%%

%%% do not modify the following VLDB block %%
%%% VLDB block start %%%
\ifdefempty{\vldbavailabilityurl}{}{
\vspace{.3cm}
\begingroup\small\noindent\raggedright\textbf{PVLDB Artifact Availability:}\\
The source code, data, and/or other artifacts have been made available at \url{\vldbavailabilityurl}.
\endgroup
}
%%% VLDB block end %%%

%%% Start of paper content %%%
% Absatz warum automatische Methoden wichtig sind.
% More Data makes it impossible for humans to manually find dependencies.
% Data Growth over years
\section{Complexity of (p)ind discovery}
The amount of data that is being generated is growing constantly and at an ever increasing pace. All digital data is estimated to double approximately every two years \cite{gantz2012digital}. Throughout this research, we will focus on structured data, a subset of all digital data, which refers to a type of data that is organized and formatted in a consistent manner, allowing efficient search, retrieval, and analysis \cite{gryz1998query}. More precisely, we will examine relational data, which is a type of structured data organized into relations. Each relation holds a set of attributes that hold a collection of values. Examples of efforts to collect relational data from the internet are the \textit{Web Table Corpora}\footnote{\href{https://webdatacommons.org/webtables/}{webdatacommons.org/webtables/} (Last Access: 30/06/2024)} or \textit{Wikitables}\footnote{\href{http://websail-fe.cs.northwestern.edu/TabEL/}{websail-fe.cs.northwestern.edu/TabEL/}  (Last Access: 30/06/2024)}. These initiatives gather relations, but do not offer the insights that can be derived from the data. Relational data typically comes from a variety of sources, including government agencies, businesses, and research institutions.

Among the most fundamental constructs in relational data profiling are inclusion dependencies (INDs), a super set of foreign key relationships \cite{casanova1982inclusion}. They assert that the values within a set of attributes from one relation form a subset of the values within another set of attributes from a possibly different relation. Foreign keys are a crucial aspect of relational databases as they help define relationships between relations, maintain referential integrity, prevent errors, and improve the performance of operations. They guarantee that every entry in one table matches a valid entry in another table, thus enhancing the consistency and precision of the database. Although foreign keys are not obligatory, they are instrumental in defining explicit relationships between tables and ensuring data validation during row insertion, updating, or deletion. By linking data between tables, new insights can be extracted and previously hidden knowledge might get revealed. In today's economy, data profiling and therefore also the discovery of foreign keys (and further INDs), are a necessity which, if done by human experts, is connected to huge cost \cite{halevy2006data}.\\

Within the rapidly evolving domain of data management and analytics, the precise representation and comprehension of interrelationships within datasets constitute a fundamental challenge. INDs encapsulate hierarchical links between attributes, thus playing a pivotal role in ensuring data integrity and normalization \cite{casanova1982inclusion}. The identification of these dependencies has significant implications for a multitude of applications, including database design \cite{levene2000justification}, query optimization \cite{gryz1998query}, and data quality assurance \cite{fan2008dependencies}. As the volume and complexity of data continues to grow, there is a simultaneously growing demand for sophisticated methodologies and tools capable of extracting the inherent inclusion dependencies within datasets.

\begin{table}[!t]
\parbox{.42\linewidth}{
\resizebox{0.42\columnwidth}{!}{%
\centering
\begin{tabular}{llll}
\hline
\footnotesize \textbf{Name}& \footnotesize \textbf{Id}& \footnotesize \textbf{Born}&\footnotesize \textbf{Died}\\
\hline
Sophie & 4 & 16 & 43 \\
Hannah & 1 & 8 & 92 \\
Angela & 2 & 6 & 30 \\
Angela & 3 & 7 & 29 \\
Sophie & 3 & 11 & 40 \\
Sophie & 2 & 6 & 10 \\
Angela & 1 & 0 & 12 \\
Sophie & 1 & 2 & 52 \\
\end{tabular}
}%
\caption{Person}
\label{tab:person}
}
\hfill
\parbox{.52\linewidth}{
\resizebox{0.52\columnwidth}{!}{%
\centering
\begin{tabular}{lll}
\hline
\footnotesize \textbf{Name}& \footnotesize \textbf{P\_Name}& \footnotesize \textbf{P\_Id}\\
\hline
Vale Villa & Angela & 2 \\
Serpent Spire & Hanna & 1 \\
Thunder Tower & Sophie & 3 \\
Rural Retreat & Angela & 1 \\
Aurora Haven & Sophie & 2 \\
Mirage Mansion & Angela & 3 \\
\end{tabular}
}%
\caption{House}
\label{tab:house}
}
\end{table}

In Tables \ref{tab:person} and \ref{tab:house} we find a simplistic example of relational data. \textit{Person} and \textit{House} are supposed to represent data collected by some community regarding the villagers and houses of that community. \textit{Person} carries the villagers names, their id and the years they where born and died in. The id column is used to distinguish between villagers with the same name. \textit{House} holds information on the buildings name and its owner which is stored in \textit{P\_Name} (person name) and \textit{P\_Id} (persons id). Notice how there seems to be a spelling mistake in \textit{Hanna} for the \textit{Serpent Spire} building. Classic INDs will find the relations $$\textit{House}[\textit{PPost}]\subseteq \textit{Person}[\textit{Post}] \textit{House}[\textit{PPost}]$$ while pINDs would be able to discover $$\textit{Person}[\textit{Name}, \textit{Post}] \subseteq_{0.85} \textit{House}[\textit{PName}, \textit{PPost}]$$ at a threshold of $85\%$. This simplistic example shows that insights generated through different partial thresholds are not merely academic, they have practical implications for companies and governments alike. If a partial inclusion dependency is found at a threshold of $99\%$, organizations could use this information to check for impurities in the given attributes. In the following, we will first discuss the foundations (Section~\ref{sec:foundations}) of pIND and define which properties of IND discovery hold for the discovery of pIND. We go over existing state-of-the-art algorithms and expand them to enable pIND discovery (Section \ref{sec:algo_partial}). These algorithms will be used as a reference point for later evaluation. Section \ref{sec:spind} will introduce our own proposal \textit{SPIND}. It solves major disadvantages of \textit{SPIDER} \cite{bauckmann2006efficiently} and \textit{BINDER} \cite{papenbrock2015divide} while offering stronger performances and more flexibility in both unary and nary settings. We then analyze our experimental results and examine pINDs within real-world datasets (Section \ref{sec:eval}). Finally, related research is discussed in Section \ref{sec:rel_work}.

% This document should discuss passed approaches focusing on pros and cons

\section{Related Work}\label{sec:rel_work}

Inclusion dependencies (INDs) are a highly influential concept in both database research and practice, with a wide range of contributions and applications. The introduction already provided some insight into their diverse application areas of INDs. In this section, we will focus on the key achievements related to the implication problem of INDs. We will go over different algorithms and discuss their unique features. \\

In 1981 INDs started as a general notation of referential integrity, which was already a well established concept back then \cite{date1981referential}. Casanova et al. presented a paper on the inference rules of INDs\cite{casanova1982inclusion}. Three axioms where introduced: \textit{reflexivity}, \textit{transitivity} and \textit{projection and permutation}. The application of these rules to partial INDs will later be discussed in detail (Section \ref{theo:pInd}). The paper further proofed that the discovery of INDs is PSPACE-complete if there is no limit on the size of inclusions. Publications typically fall into three groups of algorithms, foreign key discovery algorithms, unary IND discovery and n-nary IND discovery \cite{papenbrock2017data}. \\

In 1995 Bell and Brockhausen \cite{bell1995discovery} propose a graph-based approach to represent the relationships between attributes, allowing for a more efficient exploration of the search space. The algorithm is initiated with a directed graph, wherein all possible edges, which could not be pruned by statistical measures, are included. A directed edge in the graph represents an inclusion dependency, which is read as the edge from $A_i$ to $A_j$ ($A_i \rightarrow A_j$) represents the IND $A_i \subseteq A_j$. It then proceeds to remove those edges that failed the IND check. To determine the validity of an edge, the algorithm checks for transitivity, which enables it to answer whether a dependency could exist between two variables based on their relationships with a third variable, which was tested previously. If it is ascertained that a dependency is impossible, the algorithm skips the test and directly removes the given edge, thereby reducing the overall computational cost. The approach presented by Bell and Brockhausen for unary IND discovery has both reusable aspects and downsides. The algorithm's candidate generation technique, which uses data statistics such as data types and min-max values, can be reused in other discovery algorithms. This preprocessing step reduces the number of candidates that need to be validated and further reduce the storage overhead needed to store candidates. However, the validation technique used in the algorithm, which relies on SQL join-statements and requires accessing the data on disk, is infeasible for larger candidate sets. This limits the scalability of the approach and makes it less practical for large-scale data sets. Additionally, the need to store the data in a database and access it on disk for validation can add to the computational cost and time required for the discovery process.\\

The \textit{SPIDER} algorithm is a disk-backed, all-column sort-merge join with early termination used for the discovery of inclusion dependencies \cite{bauckmann2006efficiently}. It sorts the values of each attribute, removes duplicate values, and writes the results to disk in the first phase. In the second phase, it performs the actual inclusion dependency discovery by using a pointer for each file and validating all candidates at the same time. A major advantage is, that in this setting every value only needs to be read a single time from disk, which greatly reduces the I/O bottleneck. The Spider algorithm has been the subject of experimental evaluation and is considered one of the efficient techniques for unary IND discovery. Still, there are drawback if the data set is too big to be sorted in main memory or if the number of simultaneously open files allowed by the OS system is reached \cite{papenbrock2015divide}. \\

In 2009 Bauckmann et al. also proposed a partialized version of $SPIDER$ in a section of the same paper. The authors found that there where surprisingly many partial inclusions (under a 5\% threshold) in their test data sets. To find partial inclusion dependencies, they first count how many distinct violations are present and in a second step consider the amount of duplicates for not included values. This means their algorithm does not immediately stop once a single validation has been found but only after an added counter surpasses a given threshold. The paper is not particular clear on how the number of duplicates is stored/retrieved and additionally does not analyse the computational effect of these changes and with the original source code being lost, this a approach can only be verified using a best guess approach.\\

The \textit{SAWFISH} algorithm \cite{kaminsky2023discovering}, published in 2023, is designed for identifying similarity inclusion dependencies (sINDs) within datasets, introducing a novel perspective on inclusion dependencies (INDs). While traditional INDs assume error-free data, \textit{SAWFISH} incorporates a similarity measure to accommodate minor errors like typos. Given a similarity threshold $\omega$ a sIND is valid if and only if for all tuples in the left hand side there exists a tuple in the right hand side which has at least a similarity of $\omega$ under a set similarity measure. The authors used the edit distance as well as the normalized edit distance. Through preprocessing, metadata generation, and a sliding-window approach, \textit{SAWFISH} successfully identifies and validates sIND candidates using an inverted index, providing a valuable tool for database applications despite dirty data challenges. \\



\chapter{Definitions}

To ensure that the content of this thesis is readable by both experts and interested people we need to formulate notations and definitions. These will reappear multiple times within the thesis and they are needed to formulate precise observations and draw conclusions.

\begin{definition}[Attributes]\label{def:attributes}
    An attribute $\mathbb{X}$ is a collection of values. The values can have different data types. Every attribute has a fixed length with is equal to the number of not necessarily unique values it contains. A collection of attributes $\mathbb{C} = \{\mathbb{X}_1, \mathbb{X}_2, ... \}$ is a set where all attributes have to be of equal length.
\end{definition}

\begin{definition}[Schemas]\label{def:schema}
    Define schemas
\end{definition}


Inclusion Dependencies (INDs) represent a fundamental concept denoting formal relationships between attributes in a database schema. An Inclusion Dependency specifies that the values within one set of attributes are inherently included within the values of another set of attributes.

\begin{definition}[Inclusion Dependencies]\label{def:inds}
    An IND is written as $\mathbb{S}_1[\mathbb{C}_1] \subseteq \mathbb{S}_2[\mathbb{C}_2]$ where $\mathbb{C}_1$ and $\mathbb{C}_2$ are collections of attributes of equal size and $\mathbb{S}_1$ and $\mathbb{S}_2$ are schemes. An IND is valid, if and only if, for each tuple in $\mathbb{S}_1$, the values of attributes within $\mathbb{C}_1$ are also found within the corresponding attributes in $\mathbb{C}_2$ in $\mathbb{S}_2$.
\end{definition}

The complexity of discovering inclusion dependencies forms one of the hardest problems in computer science. More precisely, the discovery of all inclusion dependencies is W[3]-hard \cite{blasius2017parameterized}. %TODO explain W3

The number of possible candidates for each attribute size can be calculated. Notice that the formula assumes that all IND the the layers before where valid. In natural language we search for the number of attribute combinations where each attribute is present at most once, allowing all permutations.
\begin{definition}[Candidate Space]\label{def:candidates}
    Let $\alpha$ be the fixed integer size of all possible $\mathbb{C}_i$. Let $m$ be the number of attributes. Let $k$ be the number of possible candidates ( $\mathbb{C}_i \subseteq \mathbb{C}_j$ where $i \not = j$ given $\alpha$.
    $$
        k = \binom{\alpha}{n} \cdot \binom{\alpha}{n-\alpha} \cdot 2 \cdot \alpha!.
    $$
    This formula holds if $\alpha \leq \lfloor \frac{n}{2} \rfloor$ else $k$ is $0$.
\end{definition}
% TODO Tabellen mit unterschiedlichen längen berücksichtigen.

\begin{definition}[Partial Inclusion Dependencies]\label{def:pinds}
    A partial inclusion dependency (pIND) is written as $\mathbb{S}_1[\mathbb{C}_1] \subseteq_{\rho} \mathbb{S}_2[\mathbb{C}_2]$ where $\rho \in (0, 1]$ and the reaming notation is analog to \ref{def:inds}. Here, the $\rho$ interval is not including $0$ since this would mean everything is a pIND of everything else, which is a trivial case. Further this notation refers to lists of tuples and takes the cardinality of duplicates into consideration. For the pIND $\mathbb{S}_1[\mathbb{C}_1] \subseteq_{\rho} \mathbb{S}_2[\mathbb{C}_2]$ to be valid
    $$
        \frac{|\mathbb{S}_1[\mathbb{C}_1] \cap \mathbb{S}_2[\mathbb{C}_2]|}
            {|\mathbb{S}_1[\mathbb{C}_1]|} \geq \rho
    $$
    needs to be true.
\end{definition}

In the proposed algorithms there is the option of considering duplicate cardinalities. If not explicitly mentioned otherwise this thesis always refers to partial inclusions that consider duplicate cardinality.

\begin{restatable}[Partial Inclusion Dependency Properties]{theorem}{pInd}\label{theo:pInd}
    Like inclusion dependencies, partial inclusion dependencies also full fill the reflexive rule. For any $\rho \in (0, 1]$ the partial inclusion dependency $\mathbb{S}_i[\mathbb{C}_j] \subseteq_{\rho} \mathbb{S}_i[\mathbb{C}_j]$ is valid.
    \begin{align*}
        \frac{|\mathbb{S}_i[\mathbb{C}_j] \cap \mathbb{S}_i[\mathbb{C}_j]|}
            {|\mathbb{S}_i[\mathbb{C}_j]|} & \geq \rho \\
        \frac{|\mathbb{S}_i[\mathbb{C}_j]|}
            {|\mathbb{S}_i[\mathbb{C}_j]|} & \geq \rho \\
            1 & \geq \rho
     \end{align*}
     Since $\rho$ is upper bounded by $1$ the last statement will always be true. \\

     \noindent Contrary to INDs, pINDs do not generally respect transitivity if $\rho < 1$. We will proof this claim by contradiction. Assume $\mathbb{X}_1 = [1, 2, ..., 100], \mathbb{X}_2 = [2, ..., 1000], \mathbb{X}_3 = [10, 11, ..., 1000],$. If transitivity for any $\rho$ would hold, that we should find that for $\rho \in (0, 1]$ where $\mathbb{X}_1 \subseteq_\rho \mathbb{X}_2, \mathbb{X}_2 \subseteq_\rho \mathbb{X}_3$ are valid, $\mathbb{X}_1 \subseteq_\rho \mathbb{X}_3$ also needs to be valid. For the given example, if $\rho = 0.95$, we find a contradiction. \\

     \noindent Lastly, INDs and also pINDs respect projection. We will now outline a proof for this claim. Consider the attributes $\mathbb{X}_1, \mathbb{X}_2, \mathbb{X}_3, \mathbb{X}_4$ where $\mathbb{X}_1$ and $\mathbb{X}_2$ are in the same relation and $\mathbb{X}_3$ and $\mathbb{X}_4$ are in the same relation. Assume $\mathbb{X}_1, \mathbb{X}_2 \subseteq_\rho \mathbb{X}_3 \mathbb{X}_4$ is valid for some $\rho \in (0, 1]$. If projection holds, this implies that $\mathbb{X}_1 \subseteq_\rho \mathbb{X}_3$ and $\mathbb{X}_2 \subseteq_\rho \mathbb{X}_4$ have to be valid as well. If we now only consider the portion (with reduced size $\rho\%$) which satisfies $\mathbb{X}_1, \mathbb{X}_2 \subseteq_1 \mathbb{X}_3 \mathbb{X}_4$ we can use the known properties for INDs and conclude that for at least $\rho\%$ $\mathbb{X}_1 \subseteq_1 \mathbb{X}_3$ and $\mathbb{X}_2 \subseteq_1 \mathbb{X}_4$ has to be valid. This also directly implies that $\mathbb{X}_1 \subseteq_\rho \mathbb{X}_3$ and $\mathbb{X}_2 \subseteq_\rho \mathbb{X}_4$ will be true if the remaining $1-\rho$ values are added again. \\
     This property is very important for search space pruning, which is the single most important task for (p)IND discovery \cite{liu2010discover}.
\end{restatable}

\begin{restatable}[Partial Inclusion Dependencies]{example}{pInd}\label{exmp:pInd}
    Let us consider the two attributes $\mathbb{X}_1 = [1, 2, 3, 4]$ and $\mathbb{X}_2 = [1, 1, 2, 3]$. First we can conclude, that $\mathbb{X}_2 \subseteq \mathbb{X}_1$ if an inclusion dependency, since all values present in $\mathbb{X}_2$ also occur in $\mathbb{X}_1$. This also directly causes $\mathbb{X}_2 \subseteq_{1.0} \mathbb{X}_1$ to be a valid pIND. We will now calculate the maximal partial thresholds. Like inclusion dependencies, partial inclusion dependencies are not symmetrical, this requires us to perform two calculations. We already discovered $\mathbb{X}_2 \subseteq_{1.0} \mathbb{X}_1$, which implies the maximal threshold is $1$. Let us use the proposed formula for the other direction.
    \begin{align*}
        \frac{|\mathbb{X}_1 \cap \mathbb{X}_2|}
            {|\mathbb{X}_1|} & \geq \rho \\
        \frac{|[1, 2, 3, 4] \cap [1, 2, 3]|}
            {|[1, 2, 3, 4]|} & \geq \rho \\ 
        \frac{|[1, 2, 3]|}
            {|[1, 2, 3, 4]|} & \geq \rho \\
        \frac{3}{4} & \geq \rho. \\ 
    \end{align*}
    We have found that when the threshold $\rho$ is less or equal to $0.75$, $\mathbb{X}_1 \subseteq_{\rho} \mathbb{X}_2$ is valid.
\end{restatable}

The complexity of discovering partial inclusion dependencies is inherit form the base problem. Since the complexity is calculated under a worst case assumption, it does not change when switching into the partial setting. We will later on discuss in detail how the time complexity behaves under varying thresholds (Section XX).  
% Real world examples for pIND application



% TODOs:
% properly include citations

Relational databases can include \textit{NULL} or \textit{EMPTY} values.
The allowance of such values can originate from my different sources like historic expansions of a database, missing or lost data, unknown values or entries which simply can not be filled with any concrete value. For the treatment of these values there are two main interpretations. %cite https://dl.acm.org/doi/pdf/10.1145/582095.582123  
Entries which are undefined and not relevant, and entires which are missing even tho they are relevant.Using these interpretations we will now define and discuss different approaches on how \textit{NULL} entries are treated. None of the approaches is less valid than another, but one needs to choose a treatment.

\subsection*{The \textit{unknown} interpretation}
In this interpretation the truth value of $x = y$ where $x$ or $y$ or both are \textit{NULL} is considered as unknown \cite{codd1979extending}.
It is not true or false, but rather lies under the assumption that the \textit{NULL} entry could be filled using some,
possibly distinct value. This interpretation of \textit{NULL} entries causes inclusion or equality operations to also
return neither true or false, but also an \textit{unknown} truth statement.
Our search for pINDs expects a static data input and would need to be re-run in the event of new or updated entries.
Therefor the \textit{unknown} interpretation does not provide us with usable inforation. We would rather have the
expert user decide between one of the following interpretations.

\subsection*{The \textit{subset} interpretation}
This interpretation is similar to an interpretation known as \textit{more informative tuples} \cite{zaniolo1982database}.
A \textit{more informative tuple} relation is a relation between two tuples $t_1$ and $t_2$ with attributes $A$ and $B$
where for every $a \in A$ where $t_1[a] \not = NULL$, $a \in B$ and $t_1[a] = t_2[a]$ are valid. Under this notation,
$t_2$ carries the entire inforation of $t_1$, but potentially contains more inforation for the $NULL$ entries of $t_1$.

For pIND discovery we now consider a \textit{subset} interpretation. Given two relational instances $r_1$ and $r_2$ with attribute sets
$X \in r_1$ and $Y \in r_2$ where $|X| = |Y|$. We define $r_1[X] \subseteq_\rho r_2[Y]$ as valid if there are at least $\lfloor |r_1[X]| \cdot \rho \rfloor$
tuples in $r_1[X]$ for which a \textit{more informative tuple} exists in $r_2[Y]$.

\subsection*{The \textit{foreign} interpretation}
A slight variation of the \textit{subset} interpretation. The difference being the added constraint that the referenced side is not allowed to
carry \textit{NULL} values. The motivation for this interpretation is the detection of foreign keys.
% TODO: Write how this only is a difference in the unary discovery, since the n-ary generation will automatically catch this constraint

\subsection*{The \textit{equality} interpretation}
This interpretation could be considered a somewhat optimistic version of the \textit{unknown} interpretation in a sense that we consider all $NULL$ entries as being equal. The result is equal to the subset approach, if we consider $NULL$ as just another explicit value. Logically this interpretation should be chosen if $NULL$ implicitly defines an value which can not be expressed explicitly (e.g. infinity in a numeric column), but still holds meaning and is expected in the referenced for a valid pIND.

\subsection*{The \textit{inequality} interpretation}
The pessimistic version of the \textit{unknown} interpretation. Here we consider every $NULL$ entry as an distinct value. Therefor the comparison between two entries $x$ and $y$ where one or both are $NULL$ always yields false. We are in a setting where $NULL$ entires are considered not yet filled and we choose a worst case scenario where the replacement values are all distinct form each other and the relation itself. In a non-partial setting this would restrict an attribute with at least one $NULL$ entry from forming any INDs. Applying a partial setting, there is still a chance, that attributes with only a few $NULL$ entries are able to form pINDs.

This interpretation is equal to the certain world concept. No matter how the $NULL$ values would be set, the found pINDs will stay valid.

\subsection*{Possible World}
In a possible world, all pINDs that could become valid under some $NULL$ replacement, are considered valid. This creates a whole new complexity layer when trying to choose the optimal $NULL$ replacements. If a referenced attribute has $n$ $NULL$ entries, we could avoid $n$ violations. If we do not care about deduplicates, this is easy to implementation, since we can increase the possible violations of a candidate by the number of $NULL$ in the referenced attribute. Should be care about deduplicates, this situation becomes more difficult. The optimal way of replacing $NULL$ would be to pick the $n$ values in the dependant attribute with the most occurrences which do not appear in the referenced. That way we avoid the highest number of violations. For this to work, we would need to keep track of the top $n$ highest occurring values, which create a violation. Since $n$ might be very big, we might not be able to store a priority queue or something similar in main memory, and would there for need to store these values on disk.

The described problems are certainly solvable, but are out of scope for the thesis. For this reason, the algorithms do not offer pIND discovery in a possible world setting.

\section{Partializing IND Algorithms}
To enable partial IND discovery for existing algorithms, we need to adjust their mechanism. These adjustment may be more or less complex, depending on the structure of the algorithm. We have chosen to partialize \textit{SPIDER} and \textit{BINDER} to understand how complex such adjustments are and how the runtime differs from the non-partial variants.
\section{Parializing SPIDER}
The \textit{SPIDER} Algorithm \cite{bauckmann2006efficiently} was devised for the detection of unary Inclusion Dependencies (INDs), employing an all-column sort merge join technique. It operates in two main phases. Initially, it sorts each attribute and stores the sorted values in one file per attribute. Subsequently, it conducts a $k$-way merge while simultaneously validating the candidates. \\

\noindent To partially extend the \textit{SPIDER} algorithm, it becomes essential to monitor the frequency of each value alongside the total count of unique values. Bauckmann et al. have already proposed an adjusted algorithm capable of uncovering partial Inclusion Dependencies (pINDs). Their suggestion involves integrating a counter to monitor violations, invalidating candidates when the violation count exceeds a threshold. This approach is effective when duplicates are disregarded. However, if duplication distribution is of concern, the authors merely suggest retrieving such information from a database, as values are deduplicated during sorting. While this is valid, it remains unclear how occurrences are managed thereafter. Assuming the occurrences of all values surpass the capacity of main memory, querying the database for each value could becomes necessary, which presents a massive computational overhead. This thesis proposal will introduce a method capable of handling both a \textit{duplicateAware} and a \textit{duplicateUnaware} setting in a unified manner.

\subsection{Existing Code}
The original \textit{SPIDER} paper does not have a direct linkage to the source code used for the experiments. For the work conducted by Dürsch et al. on the comparison of multiple IND discovery algorithms\cite{dursch2019inclusion}, they published an implementation of \textit{SPIDER} through GitHub\footnote{\href{https://github.com/HPI-Information-Systems/inclusion-dependency-algorithms}{github.com/HPI-Information-Systems/inclusion-dependency-algorithms}}. This implementation will be referred to as the Metanome implementation. There is other research which proposes ideas to increase the performance of $SPIDER$, which mostly safes time by discussing the underlying data structures in detail and moving to a C++ based implementation \cite{smirnov2023fast}. They conclude that a speed-up of up to 5 times is possible through memory efficient value storing and a changed approach on value sorting. Instead of sorting during reading, as the Metanome implementation does it, they read the values to vectors and utilize sorting these vectors in parallel, if the main memory filled up or the end of the input is reached.

\subsection{Sorting Adjustments}
The original implementation uses a $SortedSet$ as its key structure during the attribute processing. Every value is put into the $SortedSet$ in the order of its occurrence. If at some point, the main memory of the executing machine surpasses a set threshold, the values a written to disk. This process is called spilling. We save a (sorted) subset of the data to free main memory and in the end merge the sorted chunks together. Choosing a $SortedSet$ holds the advantages, that the values are guaranteed to be sorted. Insertion, deletion or containment checks all have an $log(n)$ complexity, where $n$ represents the number of elements. Internally a self balancing red-black is used to guarantee the operation complexity under any values. A further advantage is, that a $SortedSet$ already deduplicates the values, which yield a unique set of keys to which we can add the counts during reading with very little overhead.

\noindent \newline While this seems to be the perfect structure, the experimental results have shown much room for improvement. It has proofed to be much more efficient to store the values in a hash based structure, like a HashMap, to deduplicate entries and keep track of occurrences and only sort the key set of the map lazily before it needs to be spilled to disk. A main downside to an implementation based on a $SortedSet$ is, that the rebalancing of the red-black tree can get fairly expensive. Inserting 10 million randomly shuffled numbers into a $SortedSet$ if 4-5 times slower than inserting the same amount into a HashMap and sorting the keys afterwards \footnote{\href{https://github.com/Jakob-L-M/partial-inclusion-dependencies/blob/main/experiments/src/DataStructures.java}{Source file of experiment available through GitHub.}}. These observations concluded in the choice of HashMaps that are sorted lazily when its actually needed instead of $TreeSet$.

\noindent \newline We have decided to structure the sorted files by writing the value in one line and carrying the number of occurrences in to directly following line. This format is easy to parse and does not required additional string operations like a two column encoding would.


\subsection{Validation Adjustments}
Before the validation begins, we are aware of the number of unique and total values of every attribute. This information is used to calculate the thresholds while respecting the duplication handling mode.

\noindent \newline The validation of \textit{pSPIDER} and \textit{SPIDER} are practically equal. The only difference begin, that \textit{pSPIDER} checks if the number of violations has surpassed a threshold while \textit{SPIDER} prunes immediately once the first violation occurred.
\section{Partial BINDER}
\textit{BINDER} is a IND discovery algorithm that uses a divide-and-conquer approach to efficiently find unary and nary INDs \cite{papenbrock2015divide}. It was shown that \textit{BINDER} outperforms other exact, non-distributed state-of-the-art algorithms in both unary and nary settings \cite{dursch2019inclusion}. Due to its strong performance, we also offer an adapted version of \textit{BINDER} which can handle partial IND discovery called \textit{pBINDER}. Our findings demonstrate that the discovery of pINDs does not significantly increase computational time. Furthermore, due to an optimized candidate generation, \textit{pBINDER} shows enhanced performance relative to \textit{BINDER} for nary pIND discovery.

\subsubsection{\textbf{Existing Code.}}
The authors supplied the original \textit{BINDER} source code, which will serve as an additional reference alongside \textit{SPIDER}. The partial version, \textit{pBINDER}, closely resembles \textit{BINDER} except for updates to libraries, refactoring, improved candidate generation (see Section \ref{sec:candidate_gen}), and subsequent validation modifications. 

\subsubsection{\textbf{Validation Adjustments.}}
The Validator of \textit{BINDER} handles the pruning of IND candidates. Initially, the algorithm assumes that all candidates are valid. Whenever we find a conflicting value, which is a value that is present on the dependent side but not on the referenced side of a candidate, the validator removes the given candidate. To consider partial INDs we need to expand \textit{BINDER} such that the validator keeps track of the number of violations and only removes a candidate if more violations than a given threshold occurred. The candidate violations are initialized analogously to the initialization of \textit{SPIDER}.

Note that \textit{pBINDER} only supports a \textit{duplicateAware} setting. \linebreak \textit{BINDER's} efficiency lies in a divide and conquer approach, where attributes are split into buckets using hash functions. For example, with $n$ buckets, we use \textit{hashCode} $\% \: n$ as the separation function. During validation, buckets are iterated in some order, and if an attribute is no longer in any candidate, we skip loading its bucket to save resources. However, \textit{BINDER} cannot count distinct values if not all buckets are loaded, especially before validation starts. Implementing \textit{duplicateUnaware} pIND discovery into \textit{pBINDER} would divert the algorithm too much from the original algorithm to a point where discussing the newly introduced complexity of pIND discovery would be unreasonable.

\section{SPIND Overview}

We will now discuss our proposed algorithm $SPIND$, which stands for \textbf{S}calable \textbf{P}artial \textbf{IN}clusion \textbf{D}ependency discovery. Further the german word $Spind$ is a special kind of closet and often multiple "Spinds" are placed next to each other. This is a metaphor to the algorithms procedure. Every input relation will be transformed to a "Spind" of sorted values with connected attributes (attribute combinations), which is surely bigger than a $BINDER$ bucket, but there are far fewer "Spinds" than $BINDER$ would create buckets.

\noindent \\ \textit{SPIND} solves multiple problems which no other algorithm solves. The architecture does not need to monitor main memory consumption or estimate the size of some objects. \textit{BINDER} uses heuristics to estimate if the content of a file can be loaded without overflowing the available main memory. In some rare cases this heuristic can fail leading to an execution failure. Another issue is the creation of many files. On datasets with a lot of pINDs algorithms like \textit{BINDER} or \textit{SPIDER} will create one file per attribute (combination) regardless of the dataset size. This can lead to millions of files being created, which total a few megabytes. On top of the problems that \textit{SPIND} solves, it also outperforms \textit{BINDER} and \textit{SPIDER} computationally by a factor of up to multiple magnitudes.

\begin{figure}[h]
    \centering
    \includegraphics[width=0.49\textwidth]{figures/SPIND.pdf}
    \caption{Conceptual overview of \textit{SPIND}}
    \label{fig:spind}
\end{figure}

\subsection{Chunking the input relations}
Modern CPUs feature multiple cores capable of executing tasks concurrently. Previous studies have largely overlooked the potential of multi-threading. In contrast, $SPIND$ will leverage multi-threading extensively to maximize hardware utilization. Achieving this objective necessitates the identification of independent tasks capable of being processed in parallel.

\noindent \\ The execution starts by fetching some very basic information about the input relations. For each relational instance (input table) $SPIND$ will extract the header column (if present) and store the number of unary attributes (columns) each table has. Using a constant $CHUNK\_SIZE$ which is set by the user, we spilt each relation into somewhat equal parts. Each chunk will consist of at most $\lfloor \frac{CHUNK\_SIZE}{\# cloumns \: in \: relation} \rfloor$ rows. The complexity of processing a relation directly correlates to the number of total values in that relation. While this may be an oversimplification since there are many more factors, like the distribution of duplicate values, the raw size is easy to modify and a heuristic that can be applied without any specific dataset knowledge. Each of the resulting chunks is associated with exactly one relation and carries a subset of that relations rows.

\noindent \\ Chunking is done exactly once at the very start and is not repeated for n-ary layers. We reuse the same chunks in every layer of the n-ary pIND discovery.
% TODO Link section discussing chunking effectiveness.

\noindent \\ Since $SPIND$ almost always operates on the relation layer, a hash based partitioning, similar to $BINDER$, is not feasible. For the validation, we need to descend to the attribute layer and there we need to know which attributes (attribute combinations) share at least one value. A partitioning would therefore be required to split the dataset into $n$ groups $G_1, \dots, G_n$ such that for every tuple of values $t_i$ generate by some row, it holds that if $t_i \in G_j$ than 
\begin{itemize}
    \item[1)] $\forall \: t_k \in G \setminus G_j : t_i \cap t_k = \emptyset$
    \item[2)] and $t_i \cap t_k \not = \emptyset \: \forall t_k \in G_j$.
\end{itemize}
To find such groups we would need knowledge of all values in a relation. Even if we expand the groups row by row and merge when necessary, solving this grouping problem would create a lot of new complexity. This leads to the decision to use a plain horizontal splitting approach.

\noindent \\ At the end of the chunking phase we have a set of chucks for each relation, that all carry at most \textit{CHUNK\_SIZE} values. Larger relations have been split into more chunks while smaller relations are probably contained within a single chunk.

\subsection{Probabilistic Filtering}
Assume we would know the set $R$ of all values, which only occur in referred attributes. Such a set could be used to filter the values that would need to be sorted, merged and validated. How the filter is filled and where exactly it is used will be discussed in the sorting and validation sections. This section will explain the reason why such a filter can be used in the first way and which benefits it brings.

\noindent \\ Mathematically we require the set $R$ to hold the property $\forall \: ref \in candidates : ref \cap R = \emptyset$. Trivially, the empty set would be valid for $R$. Our goal is to find an $R$ which best approximates the union of all values present in referenced attributes (attribute combinations) without the values present in any dependant attributes.

\noindent \\ Since we may not be able to actually keep all of the values in main memory, we need a different approach of filtering the values. An important observation being, that we need to be certain, that all values which are present in at least one dependant attribute need to make it through the filter. If values which are only in dependant (or irrelevant) attributes make it through the filter, that creates more computational complexity, but will not create any incorrect results.

\noindent \\ These constraints can be full filled by building a probabilistic filter in which all values that appear in an dependant attribute at least once are stored. Such a filter can than be queried using a value to check if that value was added. If the value was added previously it will always return true. If the value was not added it may create a false positive at a rate of typically less than 5\%. There are multiple options for such filters.

\noindent \\ Research regarding probabilistic data structures focuses on two primary metrics govern research evaluation: the computational efficiency of insertion and query operations, and the required number of bits for storing individual values within the structure\cite{fan2014cuckoo}. Bloom filters are initialized with an anticipated number of elements and a target false positive rate. However, if additional values are incorporated beyond the initial expectation, Bloom filters may become oversaturated, leading to an approaching false positive rate of 1 \cite{tarkoma2011theory}. Bloom filters use about 44\% ($log_2(e)$) more space per key than the theoretical lower bound. There are probabilistic data structures which are closer to the theoretical lower bound, but these structures impose more complex strategies which are out of scope of this thesis \cite{fan2014cuckoo}.

\noindent \\ The size of the Bloom Filter is set to 100 million expected values with a false positive rate of 1\% by default. With these settings the filter will consume TODO bytes of main memory. If the available memory is very limited, the filter should be initialized with a smaller size or higher false positive rate. Table \ref{tab:filter} shows how many values are inserted into the bloom filter during u-nary pIND discovery for each dataset. In section \ref{sec:spind_val} we will discuss in detail how the values are inserted into the filter.

\begin{table}
    \begin{tabular}{c|c}
     dataset & values in filter\\ 
     \hline\hline
     data.gov & - \\ 
     \hline
     musicbrainz & - \\
    \end{tabular}
    \caption{The number of items that are inserted into the bloom filter for each dataset.} \label{tab:filter}
\end{table}

\subsubsection{Procedual Refining.} After each layer the set of values which are present in dependant attributes will be a subset or equal to the set of values of any of the preciding layers. If we construct the probabilistic filter after unary pINDs have been validated, we might can only masked out values based on the unary pINDs. At fours layer, the set of values that make up dependant attribute combinations might have shrunk significantly, but the filter could not accomondate for that. We could overcome this issue by reconstructing the filter during the validation of every layer. While this creates more computational effored it holds the potential of saving read and write operations during the sort and merge phases. To rebuild the filter in n-ary layer we use the inverse transformation (see Algorithm \ref{alg:string_transformation}) to extract the strings contained in the serialized value and compute the hashes. These steps are executed in the parallized section of the validation, while the insertion into the filter needs to be synchronized between threads.

\subsubsection{Filter evaluation.} In Figure TODO we find the execution times of all datasets (expect the most complex ones) using the bloom filter by constructing it once during u-nary discovery (\textit{Once}), using and refining the filter on every level (\textit{Refining}) and not using a probabilistic filter at all (\textit{None}). The y-Axis is relative to the longest execution time for each dataset and the displayed times are averaged over three executions for each mode. We find that using a filter that was build once or a refinded filter results in very similar execution times which are usually a bit faster than not using a filter. Since a refinded filter can save some read and write operations we will chose this version to put a smaller workload on the disk. Notice that the total execution time of \textit{Cars} is under one second for n-ary discovery, which results in the time it takes to construct the file to become quite significant. We will not give much weight to this observation since the dataset is so quick to solve anyway.


\subsection{Sorting Chunks}
% IDEA: spill based on occurrences to that point.
Our objective is to arrange each chunk in sorted order, a prerequisite for subsequent merging and validation procedures. Consistently, each chunk undergoes identical processing steps, executed concurrently.

We employ a \textit{CSVReader}\footnote{Provided by the open source library "Opencsv" \url{https://opencsv.sourceforge.net/}} for each chunk, specifying the relevant attribute combinations. Line by line processing of the chunk occurs, during which we update all associated attributes to their new value. In Section \ref{subsec:n-ary-stings} we discuss exactly how the values of n-ary attributes are assigned. For now we will take the existence of a proper string value for both unary and n-ary attributes as a given. To manage these values and their associations, we employ a hash based mapping structure where values are mapped to their respective attributes along with the frequency of occurrences, thus forming a nested map. To prevent main memory overflows, we constrain the size of this map using a constant denoted as $SORT\_SIZE$. If the map size reaches the specified threshold (which is divided by the number of threads) or when sorting is complete, the contained data is serialized to disk.

This storage process will first sort the key set of the outer map. Afterwards we iterate over the sorted entries and write two lines for each of them. The first line contains the string value, which is the entries key. Followed by the the attributes which formed the value combined with the occurrences. Thereby the second line persists the value of the entry. Figure \ref{fig:sorting} illustrates this workflow while sorting for unary pIND discovery. The Attributes are $1:$\textit{id}, $2:$\textit{postCode}, $3:$\textit{landline}. For every spilled file, we retain the path of the sorted (sub) chunk, ultimately returning a list of these paths.

A minor optimization conduct here is spilling values based on the sum of occurrences in the connected attributes. We know that in a single chunk, there are at most \textit{CHUNK\_SIZE} unique values. In n-ary layers there might be relations where the number candidates of than relation is greater than the number of columns. Therefor we can calculate that the maximal possible number of distinct values generated from a chunk is $N = \underbrace{\lfloor \frac{CHUNK\_SIZE}{\#cloumns} \rfloor}_{= \#rows} \cdot \#candidates$, since every row can produce at most one unique value for each candidate. We can now say that $\frac{N}{SORTED\_SIZE}$ is the largest possible minimum of occurrences over all present values. The proposed heuristic suggests to keep all values with more than $\frac{\#candidates \cdot CHUNK\_SIZE}{SORTED\_SIZE}$ occurrences which is greater than $\frac{N}{SORTED\_SIZE}$ in main memory and only spill those values which where not seen often. The idea being, that values which have occurred often will be more likely to occur often. Spilling these last can save complexity during merging since less candidate occurrences need to be deserialized.

\begin{figure}[h]
    \centering
    \includegraphics[width=0.45\textwidth]{figures/Sorting.pdf}
    \caption{Simplified illustration of the sorting process.}
    \label{fig:sorting}
\end{figure}

\subsection{Merging (spilled) Chunks}
We now have a bunch of files which are all sorted by themselves. The next step is to merge all files which originated from the same relation. Again, this is computed in parallel. To avoid too many files being opened at the same time the constant \textit{MERGE\_SIZE} can be used to limit the number of files which are being sorted per thread. Due to multithreading the actual number of simultaneously opened files is $\#threads \cdot (MERGE\_SIZE + 1)$. The plus one originates since we always need to open the resulting output file of a merge as well. We attach a buffered reader to each file and use a priority queue to perform a k-way merge \ref{taniar2008high}. At the end of the merge phase, we have constructed a single sorted file for each relation.

\subsection{Validation}\label{sec:spind_val}
The validation process expects a sorted file per relation. The validation process works analog to the validation performed by \textit{SPIDER}. Given the sorted files we check which attributes (attribute combinations) share a value and move the readers forward bit by bit in a way that only those readers are updated which share the same smallest value. \textit{SPIDER} uses a priority queue which stores the head value of each reader. Such a data structure enables us to calculate the smallest of the head values in $log(M)$ time, $M$ being the number of attributes (attribute combinations). Assume the complexity of pruning the connected attributes to some value is $O(k)$. Validating all values $N$ would cost $O(k \cdot N \cdot log(M))$. Pruning is a synchronous task, meaning the active candidates need to be synchronized between every prune. We will therefor try to create the attribute groups in parallel and conduct pruning in the main thread.

\noindent \\ We introduce the last variable \textit{VALIDATION\_SIZE}. It is a fixed value which states how many values are loaded at the same time for each relation. Deserialization is computed in parallel for all readers and we calculate the smallest largest value for each relation queue after they have been refilled. This largest smallest queue value is calculate at most $\lceil\frac{N}{\textit{VALIDATION\_SIZE}}\rceil$ times if all $N$ values distributed over the different relations where distinct. The value groups are then also build in parallel using a stream architecture to efficiently include the synchronized pruning. The resulting complexity is  $O(k \cdot N + \lceil M \cdot \frac{N}{\textit{VALIDATION\_SIZE}}\rceil)$. Should $M$, the number of relations, be very large, the logarithm would yield a better theoretical complexity eventually. But since $M$ can be expected to be much smaller than the \textit{VALIDATION\_SIZE} and the proposed version operates mostly in parallel, we find that this concept of loading groups in parallel decreases the compute time by a factor of up to five and especially helps when processing large datasets.



\subsection{Probabilistic Filtering}\label{subsec:prob_filter}

Suppose that we have the set $I$ containing all values that never appear in a dependent attribute. This set can be utilized to exclude values that are irrelevant for pIND discovery. There is no need to sort, merge, or validate these values. This section discusses why such a filter is useful and the advantages it offers.

Mathematically, we require the set $I$ to hold the property
$$u \cap I = \emptyset \: \forall \: u \in r_i[X],$$
where $r_i[X]$ is the left-hand side of a valid pIND.
Trivially, the empty set would be valid for $I$. Our goal is to find the largest set $I$ which can be build without significant computational overhead. Given the limitation of storing all values in main memory, an alternative method for filtering values is necessary. It is crucial to ensure that any value appearing in at least one dependent attribute passes through the filter. Although values found solely in referred or irrelevant attributes may increase computational complexity if they pass through the filter, they will not lead to incorrect results.

These constraints can be satisfied by constructing a probabilistic filter that stores all values appearing in a dependent attribute at least once. This filter can then be queried with a value to verify if it was added. If the value was previously added, the filter will always return true. If the value was not added, it can produce a false positive with a probability of typically less than 5\%. Studies on probabilistic data structures concentrate on two primary metrics that guide research assessment: the computational efficiency of insertion and query operations, and the required number of bits for storing individual values within the structure \cite{fan2014cuckoo}. Bloom filters are initialized with an anticipated number of elements and a target false positive rate. However, if additional values are incorporated beyond the initial expectation, Bloom filters may become oversaturated, leading to an approaching false positive rate of one \cite{tarkoma2011theory}. Bloom filters use about 44\% more (precisely $\log_2(e)$) space per key than the theoretical lower bound. There are probabilistic data structures which are closer to the theoretical lower bound \cite{fan2014cuckoo}, but these structures impose more complex strategies which is why we decided to choose a Bloom filter for \textit{SPIND}. The size of the Bloom Filter is set to 100 million expected values with a false positive rate of 1\% by default. With these settings, the filter consumes about 400 MB of main memory. If the memory available is highly restricted, the filter should be set up with a reduced size or an increased false positive rate. Refer to Section \ref{sec:spind_val} for details on the insertion of values into the filter.

\subsubsection{Procedural Refining.} After each layer, the values in the dependent attributes will be a subset or equal to those in preceding layers. Constructing the bloom filter after validating unary pINDs allows masking values based on unary pINDs. By the fourth layer, the dependent attribute values may have shrunk significantly, but the filter could not accommodate this. Reconstructing the filter at each layer's validation increases computational effort, but can save read and write operations during following sort and merge phases. To rebuild the filter in nary layers, we use our inverse string transformation (see Section \ref{subsec:nary-strings}) to extract strings from the serialized value and compute hashes. These steps are parallelized, while filter insertion must be synchronized between threads.


\section{Partial n-ary candidate generation}
To discover n-ary pINDs, we need to generate candidates for the next layer using the already collected information of the current layer. An iterative process wherein progressively larger candidate sets are generated and verified follows the idea of an apriori-gen algorithm \cite{agrawal1994fast}. In related research we find proposals that only use the most recent layer to generate the candidate set for the following layer \cite{papenbrock2015divide}. We will propose a refined version, which significantly decreases the number of false candidates that need to be validated.

As proven (partial) INDs respect projection. For any pIND of size $n$, all subsets of that pIND must be also valid. This property has not been used to its full potential by past research. An adjusted version for improved candidate generation needs to follow the three rule below.

Given two not necessarily different relational schema $R_a$ and $R_b$ with valid (n-1)-ary attributes $X \subset R_a$ and $Y \subset R_b$ and valid u-nary attributes $A \in R_a$ and $B \in R_b$, a valid n-ary candidate is formed if
\begin{itemize}
    \item[1)] The column index of $A$ is greater than the column index of all $x_i \in X$.
    \item[2)] $B$ is not contained in $Y$.
    \item[3)] All (n-1)-ary pINDs $R_a[(X \setminus X_i)A] \subseteq_\rho R_b[(Y \setminus Y_i)B] : i \in (1, \dots, n-1)$ must be valid.
    \item[4)] If $R_a$ and $R_b$ are the same scheme, then $X$, $A$, $Y$ and $B$ need to be pairwise disjoint.
    
\end{itemize}

Originally, $B$ was also required to be non-empty, which makes sense when treating null as a subset. In that setting an all-null attribute will create a valid n-ary candidate with all (n-1)-ary pINDs and is therefore trivial. The proposed algorithm however offer different null interpretations under which all-empty columns might become relevant. This is why we will not require $B$ to be non-empty.

We will further apply another step to the candidate generation which improves the original method by pruning candidates that are not possible during the generation without the need of the actual relation instances. We have showed that transitivity requires all subsets of a candidate to be valid for that candidate to be valid (Section \ref{theo:pInd}). In the original generation, this is only applied in a weak setting by ensuring $r_1[X] \subseteq_\rho r_2[Y]$ and $r_1[A] \subseteq_\rho r_2[B]$. Assume $|X| = 3$, we know that if $r_1[X_1, A] \subseteq_\rho r_2[Y_1, B]$ or $r_1[X_2, A] \subseteq_\rho r_2[Y_2, B]$ are not present in the current layer, then $r_1[X_1, X_2, A] \subseteq_\rho r_2[Y_1, Y_2, B]$ can not be valid. Therefor the proposed expansion is to ensure the subsets of the generated candidates are also valid.

\begin{algorithm}[hbt!]
    \caption{Subset candidate check}\label{alg:canditate_gen}
    \KwInput{$rDep, nDep, uDep, rRef, nRef, uRef$}
    \KwOutput{Whether the next level candidate is possible}

    \For{$skip$ in $1, \dots, |nDep|$}{
        dependant = $rDep[(nDep \setminus nDep_{skip}), uDep]$ \\
        referred = $rRef[(nRef \setminus nRef_{skip}), uRef]$ \\
        \If{dependant $\not \subseteq_\rho$ referred}{
            \Return False}
}
    \Return True
\end{algorithm}

Using the subset check is only sensible when generating candidates for the third or higher layer. Unary candidates are trivially generated using all available columns and the second layer is a combination of two unary pINDs which directly implies that all subsets have to be valid.

\subsection{Serializing n-ary tuples.}
Our unary discovery is based on strategies which relie on hashing and value comparison (sorting). By default, a list of values (tuple) is a non-hashable type. We could define a custom function and implement comparison functions for tuples. It would require us to use different functions for unary and n-ary discovery, even if they practically do the same. Using a string representation of tuples instead, enables us to use the unary procedures for n-ary pIND discovery, creating a simpler algorithmic structure. \textit{BINDER} also uses this strategy, where the values contained in some multi-dimensional tuple are concatenated using "\#" as a separator. The tuple $("Marburg", "35037", "06421")$ gets transformed to the string $"Marburg\#35037\#06421"$. Concatenating values in such a way can produce incorrect results since there can be multiple tuples which create the same string. We search for for a bijective transformation $f$ which accepts a tuple and creates a string. The function used by \textit{BINDER} is not bijective since the tuples $(1\#1, 1)$ and $(1, 1\#1)$ will both produce the same string. While this proves to be highly unlikely in real datasets, we should still eliminate this point of failure.
Or proposal function will take a given $k$ dimensional tuple and first conduct a length encoding for the first $k-1$ values. These lengths are concatenated using ":", but any non-numerical character would work here. We mark the end of the length encoding using "|", again the precise choice of that character is not important. After the length encoding all values are concatenated without using a delimiter. The two tuples from before would be transformed to$("1\#1", "1") \rightarrow "3|1\#11"$ and $("1", "1\#1") \rightarrow "1|11\#1"$.
To prove that our proposal is indeed bijective, we need to make sure that every string that can be produced by our function originates form exactly one input. We will construct the inverse function $f^{-1}$ and show that the chain $f circlejoin {f^-1}$ is an identity function for all tuples. The inverse function will first search for the first occurrence of "|" which marks the end of the length encoding. It will then split the second part based on the length information to extract the inputs.
Given a $k$ dimensional tuple of strings $(s_1, \dots, s_k)$, $f$ produces the string $"len(s_1):len(s_2):\dots:len(s_{k-1})|s_1 s_2 \dots s_k"$. The inverse function $f^{-1}$ can now cut the second part at the correct indices re-producing the tuple $(s_1, \dots, s_k)$. This function is independent of the characters present in the strings, since we only care about the first occurrence of "|" which is guaranteed to be the end of the length encoding due to the explained serialization procedures. Algorithm \ref{alg:string_transformation} shows how the transformation and inverse transformation can be implemented.

\begin{algorithm}[hbt!]
    \caption{N-ary Attribute String Creation}\label{alg:string_transformation}
    \KwInput{\textit{attributeValues}: The list of string values}
    \KwOutput{A String representation of all attribute values}

    \If{attributeValues.length == 1}{
        \Return \textit{attributeValues}[1]
    }

    representation = "" \\
    \tcc{Create the length encoding}
    \For{index \textbf{in} 1, \dots, attributeValues.length - 1}{
        representation.append(attributeValues[index].length) \\
        representation.append(":")
    }
    representation.append("|") \\
    \tcc{Append the actual values}
    \For{value \textbf{in} attributeValues}{
        representation.append(value)
    }
    \Return representation
\end{algorithm}

\chapter{Experimental Evaluation}\label{sec:eval}
To understand the practical performance of the proposed algorithms \textit{pSPIDER}, \textit{pBINDER} and \textit{SPIND} we use the known algorithms \textit{SPIDER} and \textit{BINDER} with their existing implementations. We will review the data sets utilized (Section \ref{subsec:datasets}) and present the execution times for all data sets across the different algorithms (Section \ref{subsec:runtime}). We will also discuss the impact of hyperparameter choices (Section \ref{subsec:hyperparameters}), the probabilistic filter (Section \ref{subsec:filter_res}), and finally examine pINDs in real-world data (Section \ref{subsec:real_pINDs}).

\chapter{Datasets}
To understand the performance of the proposed algorithms it is crucial to perform testing on a variety of data sets. For this purpose we will gather some real word data sets. Further we will create synthetic data sets that aim on edge cases to see if the performance is strongly dependent on structural assumptions.

\section{Real World Data Sets}
There are many sources for csv or tsv files online. I have decided to gather data from the US Government\footnote{\href{https://data.gov}{data.gov}}, the European Union\footnote{\href{https://data.europe.eu}{data.europe.eu}}, Kaggle\footnote{\href{https://kaggle.com}{Kaggle.com}}, Musicbrainz\footnote{\href{https://musicbrainz.org/}{musicbrainz.org}}, and Eurostat\footnote{\href{https://musicbrainz.org/}{ec.europa.eu}}. Further data set sources may be added. Related research papers sometimes discuss the origin of the used data, do not discuss the structure of the data they use \cite{papenbrock2017data,bauckmann2006efficiently, dursch2019inclusion, rostin2009machine}. In order to understand the resulting algorithm performance, we believe it is crucial to examine the data which is tested against. In this section we will discuss the data used and later try to understand why an algorithms performance may vary over different test sets.

\section{Synthetic Data Sets}
To evaluate the proposed algorithms under detailed aspects, we will generate synthetic data sets. The strategies and claims are based on \cite{jordon2022synthetic} synthetic data can be defined as \textit{data that has been generated using a purpose-built mathematical model or algorithm, with the aim of solving a (set of) data science task(s).} While we will not try to train a model with the synthetic data, it is still of great use for us, since we have absolute knowledge about the underlying structures. The decision is based on the fact that there is a lot of real word data available, since open data is a growing market which expected to grow even further \cite{EUopenData}. Synthetic data on the other hand enables us to evaluate the algorithm performances on edge cases, which we may not be able to find in the selection of real world data sets. \\

\noindent To test certain edge cases of the proposed algorithms, we will construct various edge case data sets. The \textit{SameSame} dataset consists of 32 attributes and 250.000 records. Each attribute carries the numbers 1 to 250.000 in the natural order. This means every attribute is a (partial) inclusion dependency of every other attribute. The same obviously also holds for combinations of columns. We will now calculate the expected number if (p)INDs in each layer. Since all candidates are perfect matches, the chosen threshold $\rho$ will not influence the number of pINDs. Table % TODO add ref
shows the number of candidates/pINDs for the \textit{SameSame} dateset.
% TODO calculate INDs
The edge case to test here is, how well the algorithm can understand equality relations and prune the candidate space. While this may seem like an unlikely edge case we will also investigate how often this happens in real world data sets. \\
% write about real world structures

\noindent Another source of synthetic data will be the TPC (Transaction Processing Performance Council) Benchmarks \footnote{\href{https://www.tpc.org/}{tpc.org}}. The TPC Benchmarks are a set of standardized and vendor-neutral performance benchmarks used to evaluate the processing and database capabilities of different systems. These benchmarks are designed to model various types of workloads. The TPC-E benchmark, for example, models a brokerage firm with customers who generate transactions related to trades, account inquiries, and market research, while the TPC-C benchmark is intended to model a medium complexity online transaction processing workload, patterned after an order-entry system with skewed access within individual data types/relations. Using scaling factors, a user can define the size of the synthetic database themselves. This enables us to examine the algorithm performances in a very controllable setting.



\section{Filter Evaluation} \label{subsec:filter_res}
We would like to understand the effect of using a probabilistic filter. In Section \ref{subsec:prob_filter} we discuss two versions. A filter which is built once after unary discovery and a filter which is rebuild on every nary layer. In Figure \ref{fig:filter} we find the execution times of all datasets, expect \textbf{WebTables} since we only search for unarys in that case. The graphic shows the difference in runtime when using the bloom filter by constructing it once during unary discovery (\textit{Once}), using and refining the filter at every level (\textit{Refine}) and not using a probabilistic filter at all. Not using a filter builds the baseline execution time, while the other two modes are displayed with their relative runtime to that baseline.

We find that the results vary substantially when viewing the different datasets. This observation supports the notion that we succeeded in finding a range of datasets which are structurally different from each other. Notably, \textit{TPC-H 1} and \textit{UniProt} performed better without a probabilistic filter. For \textit{UniProt} we find that the filter does decrease the execution time since the data set exclusively produces symmetrically INDs. For \textit{TPC-H 1}, we observe that the number of n-ary candidates is so limited that the time spent on construction exceeds the computational savings achieved. Employing either a filter build once or a refined filter yields nearly identical execution times, which are generally slightly quicker than not using a filter at all. Given that a refined filter can reduce some read and write operations, we opt for this version to reduce the disk workload.

\begin{figure}[t!]
    \centering
    \includegraphics[width=.6\textwidth]{figures/filter_results.pdf}
    \caption{Changes in execution time when using a (refined) probabilistic filter for nary IND discovery.}
    \label{fig:filter}
\end{figure}

\section{Real World pINDs} \label{subsec:real_pINDs}
Real-world datasets exhibit a considerable amount of pINDs, even when the identification threshold for such relationships is set to a high value (e.g., $\rho = 0.99$). Some of these pINDs are logical and significant, indicating underlying patterns or connections within the data. For example, date columns may not be perfectly contained within each other due to differences in update frequencies across data sets (found in the \textit{EU} data set). However, pINDs also occur purely by chance, without any substantial relevance. Consequently, human expertise is essential to evaluate whether a discovered pIND is truly useful, as automated detection alone cannot adequately distinguish between meaningful dependencies and coincidental ones.

\chapter{Conclusion}
Our proposed algorithm \textit{SPIND} is able to overcome the inconveniences of past algorithms while outperforming them significantly. Our findings are possible by using the changes in hardware and adapting algorithmic decisions on them. We demonstrate that I/O operations are no longer a significant bottleneck and highlight the importance of leveraging the multi-threaded capabilities of modern CPUs.

We recognize that there are unresolved topics that future research can explore. A central observation in complex datasets with a large number of nary pINDs is that the generated candidates eventually become equal or almost equal to the valid pINDs of some layer. An idea would be to jump a few layers ahead if such a thing happens, since it strongly hints towards the existence of pINDs in a (much) deeper layer. In \textit{ACNH} we find that from the seventh to the twelve layer the generated candidates were always all valid. De Marchi et al. have previously examined this type of generation \cite{de2003zigzag}, but it is necessary to determine whether this approach generally yields better performance or if it is only effective for specific data sets. 

% An additional optimization that appears to be promising involves encoding the attribute identifiers and their occurrences (integers and longs) into four- and eight-byte blocks. This method could significantly reduce the file size, thereby potentially enhancing I/O speeds considerably.

%%% End of paper content %%%

%\clearpage

\bibliographystyle{ACM-Reference-Format}
\bibliography{sample}

\end{document}
\endinput
