% VLDB template version of 2020-08-03 enhances the ACM template, version 1.7.0:
% https://www.acm.org/publications/proceedings-template
% The ACM Latex guide provides further information about the ACM template

\documentclass[sigconf, nonacm]{acmart}

%% The following content must be adapted for the final version
% paper-specific
\newcommand\vldbdoi{XX.XX/XXX.XX}
\newcommand\vldbpages{XXX-XXX}
% issue-specific
\newcommand\vldbvolume{14}
\newcommand\vldbissue{1}
\newcommand\vldbyear{2020}
% should be fine as it is
\newcommand\vldbauthors{\authors}
\newcommand\vldbtitle{\shorttitle} 
% leave empty if no availability url should be set
\newcommand\vldbavailabilityurl{https://github.com/Jakob-L-M/partial-inclusion-dependencies}
% whether page numbers should be shown or not, use 'plain' for review versions, 'empty' for camera ready
\newcommand\vldbpagestyle{plain} 

\begin{document}
\title{Partial Inclusion Dependency Discovery}

%%
%% The "author" command and its associated commands are used to define the authors and their affiliations.
\author{Jakob Leander Müller}
\affiliation{%
  \institution{Philipps-Universität Marburg}
  \streetaddress{P.O. Box 1212}
  \city{Marburg}
  \state{Germany}
  \postcode{35037}
}
\email{muelle5t@students.uni-marburg.de}
\email{me@jakob-l-m.de}

\author{Thorsten Papenbrock}
\orcid{0000-0002-1825-0097}
\affiliation{%
  \institution{The Th{\o}rv{\"a}ld Group}
  \streetaddress{1 Th{\o}rv{\"a}ld Circle}
  \city{Hekla}
  \country{Iceland}
}
\email{larst@affiliation.org}

\author{Phd Mensch}
\orcid{0000-0001-5109-3700}
\affiliation{%
  \institution{Inria Paris-Rocquencourt}
  \city{Rocquencourt}
  \country{France}
}
\email{vb@rocquencourt.com}

%%
%% The abstract is a short summary of the work to be presented in the
%% article.
\begin{abstract}
In a world where data grows exponentially and data sharing becomes more common, the need for efficient and flexible data profiling increases constantly. Past research dealing with inclusion dependencies (INDs) mostly overlooked the potential of partial INDs, meaning imperfect subsets. Such partial INDs can enable researchers and data engineers to understand databases, even if different sources do not match perfectly. Combining know strategies such as a sort-merge join and heavy parallelization we will introduce the algorithm \textit{Spind}. \textit{Spind} is not only able to find all partial INDs, it also outperforms the current state-of-the-art algorithm \textit{BINDER} in both unary and n-ary setting by multiple magnitudes.
\end{abstract}

\maketitle

%%% do not modify the following VLDB block %%
%%% VLDB block start %%%
\pagestyle{\vldbpagestyle}
\begingroup\small\noindent\raggedright\textbf{PVLDB Reference Format:}\\
\vldbauthors. \vldbtitle. PVLDB, \vldbvolume(\vldbissue): \vldbpages, \vldbyear.\\
\href{https://doi.org/\vldbdoi}{doi:\vldbdoi}
\endgroup
\begingroup
\renewcommand\thefootnote{}\footnote{\noindent
This work is licensed under the Creative Commons BY-NC-ND 4.0 International License. Visit \url{https://creativecommons.org/licenses/by-nc-nd/4.0/} to view a copy of this license. For any use beyond those covered by this license, obtain permission by emailing \href{mailto:info@vldb.org}{info@vldb.org}. Copyright is held by the owner/author(s). Publication rights licensed to the VLDB Endowment. \\
\raggedright Proceedings of the VLDB Endowment, Vol. \vldbvolume, No. \vldbissue\ %
ISSN 2150-8097. \\
\href{https://doi.org/\vldbdoi}{doi:\vldbdoi} \\
}\addtocounter{footnote}{-1}\endgroup
%%% VLDB block end %%%

%%% do not modify the following VLDB block %%
%%% VLDB block start %%%
\ifdefempty{\vldbavailabilityurl}{}{
\vspace{.3cm}
\begingroup\small\noindent\raggedright\textbf{PVLDB Artifact Availability:}\\
The source code, data, and/or other artifacts have been made available at \url{\vldbavailabilityurl}.
\endgroup
}
%%% VLDB block end %%%

%%% Start of paper content %%%
% Absatz warum automatische Methoden wichtig sind.
% More Data makes it impossible for humans to manually find dependencies.
% Data Growth over years
\section{Complexity of (p)ind discovery}
The amount of data that is being generated is growing constantly and at an ever increasing pace. All digital data is estimated to double approximately every two years \cite{gantz2012digital}. Throughout this research, we will focus on structured data, a subset of all digital data, which refers to a type of data that is organized and formatted in a consistent manner, allowing efficient search, retrieval, and analysis \cite{gryz1998query}. More precisely, we will examine relational data, which is a type of structured data organized into relations. Each relation holds a set of attributes that hold a collection of values. Examples of efforts to collect relational data from the internet are the \textit{Web Table Corpora}\footnote{\href{https://webdatacommons.org/webtables/}{webdatacommons.org/webtables/} (Last Access: 30/06/2024)} or \textit{Wikitables}\footnote{\href{http://websail-fe.cs.northwestern.edu/TabEL/}{websail-fe.cs.northwestern.edu/TabEL/}  (Last Access: 30/06/2024)}. These initiatives gather relations, but do not offer the insights that can be derived from the data. Relational data typically comes from a variety of sources, including government agencies, businesses, and research institutions.

Among the most fundamental constructs in relational data profiling are inclusion dependencies (INDs), a super set of foreign key relationships \cite{casanova1982inclusion}. They assert that the values within a set of attributes from one relation form a subset of the values within another set of attributes from a possibly different relation. Foreign keys are a crucial aspect of relational databases as they help define relationships between relations, maintain referential integrity, prevent errors, and improve the performance of operations. They guarantee that every entry in one table matches a valid entry in another table, thus enhancing the consistency and precision of the database. Although foreign keys are not obligatory, they are instrumental in defining explicit relationships between tables and ensuring data validation during row insertion, updating, or deletion. By linking data between tables, new insights can be extracted and previously hidden knowledge might get revealed. In today's economy, data profiling and therefore also the discovery of foreign keys (and further INDs), are a necessity which, if done by human experts, is connected to huge cost \cite{halevy2006data}.\\

Within the rapidly evolving domain of data management and analytics, the precise representation and comprehension of interrelationships within datasets constitute a fundamental challenge. INDs encapsulate hierarchical links between attributes, thus playing a pivotal role in ensuring data integrity and normalization \cite{casanova1982inclusion}. The identification of these dependencies has significant implications for a multitude of applications, including database design \cite{levene2000justification}, query optimization \cite{gryz1998query}, and data quality assurance \cite{fan2008dependencies}. As the volume and complexity of data continues to grow, there is a simultaneously growing demand for sophisticated methodologies and tools capable of extracting the inherent inclusion dependencies within datasets.

\begin{table}[!t]
\parbox{.42\linewidth}{
\resizebox{0.42\columnwidth}{!}{%
\centering
\begin{tabular}{llll}
\hline
\footnotesize \textbf{Name}& \footnotesize \textbf{Id}& \footnotesize \textbf{Born}&\footnotesize \textbf{Died}\\
\hline
Sophie & 4 & 16 & 43 \\
Hannah & 1 & 8 & 92 \\
Angela & 2 & 6 & 30 \\
Angela & 3 & 7 & 29 \\
Sophie & 3 & 11 & 40 \\
Sophie & 2 & 6 & 10 \\
Angela & 1 & 0 & 12 \\
Sophie & 1 & 2 & 52 \\
\end{tabular}
}%
\caption{Person}
\label{tab:person}
}
\hfill
\parbox{.52\linewidth}{
\resizebox{0.52\columnwidth}{!}{%
\centering
\begin{tabular}{lll}
\hline
\footnotesize \textbf{Name}& \footnotesize \textbf{P\_Name}& \footnotesize \textbf{P\_Id}\\
\hline
Vale Villa & Angela & 2 \\
Serpent Spire & Hanna & 1 \\
Thunder Tower & Sophie & 3 \\
Rural Retreat & Angela & 1 \\
Aurora Haven & Sophie & 2 \\
Mirage Mansion & Angela & 3 \\
\end{tabular}
}%
\caption{House}
\label{tab:house}
}
\end{table}

In Tables \ref{tab:person} and \ref{tab:house} we find a simplistic example of relational data. \textit{Person} and \textit{House} are supposed to represent data collected by some community regarding the villagers and houses of that community. \textit{Person} carries the villagers names, their id and the years they where born and died in. The id column is used to distinguish between villagers with the same name. \textit{House} holds information on the buildings name and its owner which is stored in \textit{P\_Name} (person name) and \textit{P\_Id} (persons id). Notice how there seems to be a spelling mistake in \textit{Hanna} for the \textit{Serpent Spire} building. Classic INDs will find the relations $$\textit{House}[\textit{PPost}]\subseteq \textit{Person}[\textit{Post}] \textit{House}[\textit{PPost}]$$ while pINDs would be able to discover $$\textit{Person}[\textit{Name}, \textit{Post}] \subseteq_{0.85} \textit{House}[\textit{PName}, \textit{PPost}]$$ at a threshold of $85\%$. This simplistic example shows that insights generated through different partial thresholds are not merely academic, they have practical implications for companies and governments alike. If a partial inclusion dependency is found at a threshold of $99\%$, organizations could use this information to check for impurities in the given attributes. In the following, we will first discuss the foundations (Section~\ref{sec:foundations}) of pIND and define which properties of IND discovery hold for the discovery of pIND. We go over existing state-of-the-art algorithms and expand them to enable pIND discovery (Section \ref{sec:algo_partial}). These algorithms will be used as a reference point for later evaluation. Section \ref{sec:spind} will introduce our own proposal \textit{SPIND}. It solves major disadvantages of \textit{SPIDER} \cite{bauckmann2006efficiently} and \textit{BINDER} \cite{papenbrock2015divide} while offering stronger performances and more flexibility in both unary and nary settings. We then analyze our experimental results and examine pINDs within real-world datasets (Section \ref{sec:eval}). Finally, related research is discussed in Section \ref{sec:rel_work}.

\chapter{Definitions}

To ensure that the content of this thesis is readable by both experts and interested people we need to formulate notations and definitions. These will reappear multiple times within the thesis and they are needed to formulate precise observations and draw conclusions.

\begin{definition}[Attributes]\label{def:attributes}
    An attribute $\mathbb{X}$ is a collection of values. The values can have different data types. Every attribute has a fixed length with is equal to the number of not necessarily unique values it contains. A collection of attributes $\mathbb{C} = \{\mathbb{X}_1, \mathbb{X}_2, ... \}$ is a set where all attributes have to be of equal length.
\end{definition}

\begin{definition}[Schemas]\label{def:schema}
    Define schemas
\end{definition}


Inclusion Dependencies (INDs) represent a fundamental concept denoting formal relationships between attributes in a database schema. An Inclusion Dependency specifies that the values within one set of attributes are inherently included within the values of another set of attributes.

\begin{definition}[Inclusion Dependencies]\label{def:inds}
    An IND is written as $\mathbb{S}_1[\mathbb{C}_1] \subseteq \mathbb{S}_2[\mathbb{C}_2]$ where $\mathbb{C}_1$ and $\mathbb{C}_2$ are collections of attributes of equal size and $\mathbb{S}_1$ and $\mathbb{S}_2$ are schemes. An IND is valid, if and only if, for each tuple in $\mathbb{S}_1$, the values of attributes within $\mathbb{C}_1$ are also found within the corresponding attributes in $\mathbb{C}_2$ in $\mathbb{S}_2$.
\end{definition}

The complexity of discovering inclusion dependencies forms one of the hardest problems in computer science. More precisely, the discovery of all inclusion dependencies is W[3]-hard \cite{blasius2017parameterized}. %TODO explain W3

The number of possible candidates for each attribute size can be calculated. Notice that the formula assumes that all IND the the layers before where valid. In natural language we search for the number of attribute combinations where each attribute is present at most once, allowing all permutations.
\begin{definition}[Candidate Space]\label{def:candidates}
    Let $\alpha$ be the fixed integer size of all possible $\mathbb{C}_i$. Let $m$ be the number of attributes. Let $k$ be the number of possible candidates ( $\mathbb{C}_i \subseteq \mathbb{C}_j$ where $i \not = j$ given $\alpha$.
    $$
        k = \binom{\alpha}{n} \cdot \binom{\alpha}{n-\alpha} \cdot 2 \cdot \alpha!.
    $$
    This formula holds if $\alpha \leq \lfloor \frac{n}{2} \rfloor$ else $k$ is $0$.
\end{definition}
% TODO Tabellen mit unterschiedlichen längen berücksichtigen.

\begin{definition}[Partial Inclusion Dependencies]\label{def:pinds}
    A partial inclusion dependency (pIND) is written as $\mathbb{S}_1[\mathbb{C}_1] \subseteq_{\rho} \mathbb{S}_2[\mathbb{C}_2]$ where $\rho \in (0, 1]$ and the reaming notation is analog to \ref{def:inds}. Here, the $\rho$ interval is not including $0$ since this would mean everything is a pIND of everything else, which is a trivial case. Further this notation refers to lists of tuples and takes the cardinality of duplicates into consideration. For the pIND $\mathbb{S}_1[\mathbb{C}_1] \subseteq_{\rho} \mathbb{S}_2[\mathbb{C}_2]$ to be valid
    $$
        \frac{|\mathbb{S}_1[\mathbb{C}_1] \cap \mathbb{S}_2[\mathbb{C}_2]|}
            {|\mathbb{S}_1[\mathbb{C}_1]|} \geq \rho
    $$
    needs to be true.
\end{definition}

In the proposed algorithms there is the option of considering duplicate cardinalities. If not explicitly mentioned otherwise this thesis always refers to partial inclusions that consider duplicate cardinality.

\begin{restatable}[Partial Inclusion Dependency Properties]{theorem}{pInd}\label{theo:pInd}
    Like inclusion dependencies, partial inclusion dependencies also full fill the reflexive rule. For any $\rho \in (0, 1]$ the partial inclusion dependency $\mathbb{S}_i[\mathbb{C}_j] \subseteq_{\rho} \mathbb{S}_i[\mathbb{C}_j]$ is valid.
    \begin{align*}
        \frac{|\mathbb{S}_i[\mathbb{C}_j] \cap \mathbb{S}_i[\mathbb{C}_j]|}
            {|\mathbb{S}_i[\mathbb{C}_j]|} & \geq \rho \\
        \frac{|\mathbb{S}_i[\mathbb{C}_j]|}
            {|\mathbb{S}_i[\mathbb{C}_j]|} & \geq \rho \\
            1 & \geq \rho
     \end{align*}
     Since $\rho$ is upper bounded by $1$ the last statement will always be true. \\

     \noindent Contrary to INDs, pINDs do not generally respect transitivity if $\rho < 1$. We will proof this claim by contradiction. Assume $\mathbb{X}_1 = [1, 2, ..., 100], \mathbb{X}_2 = [2, ..., 1000], \mathbb{X}_3 = [10, 11, ..., 1000],$. If transitivity for any $\rho$ would hold, that we should find that for $\rho \in (0, 1]$ where $\mathbb{X}_1 \subseteq_\rho \mathbb{X}_2, \mathbb{X}_2 \subseteq_\rho \mathbb{X}_3$ are valid, $\mathbb{X}_1 \subseteq_\rho \mathbb{X}_3$ also needs to be valid. For the given example, if $\rho = 0.95$, we find a contradiction. \\

     \noindent Lastly, INDs and also pINDs respect projection. We will now outline a proof for this claim. Consider the attributes $\mathbb{X}_1, \mathbb{X}_2, \mathbb{X}_3, \mathbb{X}_4$ where $\mathbb{X}_1$ and $\mathbb{X}_2$ are in the same relation and $\mathbb{X}_3$ and $\mathbb{X}_4$ are in the same relation. Assume $\mathbb{X}_1, \mathbb{X}_2 \subseteq_\rho \mathbb{X}_3 \mathbb{X}_4$ is valid for some $\rho \in (0, 1]$. If projection holds, this implies that $\mathbb{X}_1 \subseteq_\rho \mathbb{X}_3$ and $\mathbb{X}_2 \subseteq_\rho \mathbb{X}_4$ have to be valid as well. If we now only consider the portion (with reduced size $\rho\%$) which satisfies $\mathbb{X}_1, \mathbb{X}_2 \subseteq_1 \mathbb{X}_3 \mathbb{X}_4$ we can use the known properties for INDs and conclude that for at least $\rho\%$ $\mathbb{X}_1 \subseteq_1 \mathbb{X}_3$ and $\mathbb{X}_2 \subseteq_1 \mathbb{X}_4$ has to be valid. This also directly implies that $\mathbb{X}_1 \subseteq_\rho \mathbb{X}_3$ and $\mathbb{X}_2 \subseteq_\rho \mathbb{X}_4$ will be true if the remaining $1-\rho$ values are added again. \\
     This property is very important for search space pruning, which is the single most important task for (p)IND discovery \cite{liu2010discover}.
\end{restatable}

\begin{restatable}[Partial Inclusion Dependencies]{example}{pInd}\label{exmp:pInd}
    Let us consider the two attributes $\mathbb{X}_1 = [1, 2, 3, 4]$ and $\mathbb{X}_2 = [1, 1, 2, 3]$. First we can conclude, that $\mathbb{X}_2 \subseteq \mathbb{X}_1$ if an inclusion dependency, since all values present in $\mathbb{X}_2$ also occur in $\mathbb{X}_1$. This also directly causes $\mathbb{X}_2 \subseteq_{1.0} \mathbb{X}_1$ to be a valid pIND. We will now calculate the maximal partial thresholds. Like inclusion dependencies, partial inclusion dependencies are not symmetrical, this requires us to perform two calculations. We already discovered $\mathbb{X}_2 \subseteq_{1.0} \mathbb{X}_1$, which implies the maximal threshold is $1$. Let us use the proposed formula for the other direction.
    \begin{align*}
        \frac{|\mathbb{X}_1 \cap \mathbb{X}_2|}
            {|\mathbb{X}_1|} & \geq \rho \\
        \frac{|[1, 2, 3, 4] \cap [1, 2, 3]|}
            {|[1, 2, 3, 4]|} & \geq \rho \\ 
        \frac{|[1, 2, 3]|}
            {|[1, 2, 3, 4]|} & \geq \rho \\
        \frac{3}{4} & \geq \rho. \\ 
    \end{align*}
    We have found that when the threshold $\rho$ is less or equal to $0.75$, $\mathbb{X}_1 \subseteq_{\rho} \mathbb{X}_2$ is valid.
\end{restatable}

The complexity of discovering partial inclusion dependencies is inherit form the base problem. Since the complexity is calculated under a worst case assumption, it does not change when switching into the partial setting. We will later on discuss in detail how the time complexity behaves under varying thresholds (Section XX).  
% Real world examples for pIND application



% TODOs:
% properly include citations

Relational databases can include \textit{NULL} or \textit{EMPTY} values.
The allowance of such values can originate from my different sources like historic expansions of a database, missing or lost data, unknown values or entries which simply can not be filled with any concrete value. For the treatment of these values there are two main interpretations. %cite https://dl.acm.org/doi/pdf/10.1145/582095.582123  
Entries which are undefined and not relevant, and entires which are missing even tho they are relevant.Using these interpretations we will now define and discuss different approaches on how \textit{NULL} entries are treated. None of the approaches is less valid than another, but one needs to choose a treatment.

\subsection*{The \textit{unknown} interpretation}
In this interpretation the truth value of $x = y$ where $x$ or $y$ or both are \textit{NULL} is considered as unknown \cite{codd1979extending}.
It is not true or false, but rather lies under the assumption that the \textit{NULL} entry could be filled using some,
possibly distinct value. This interpretation of \textit{NULL} entries causes inclusion or equality operations to also
return neither true or false, but also an \textit{unknown} truth statement.
Our search for pINDs expects a static data input and would need to be re-run in the event of new or updated entries.
Therefor the \textit{unknown} interpretation does not provide us with usable inforation. We would rather have the
expert user decide between one of the following interpretations.

\subsection*{The \textit{subset} interpretation}
This interpretation is similar to an interpretation known as \textit{more informative tuples} \cite{zaniolo1982database}.
A \textit{more informative tuple} relation is a relation between two tuples $t_1$ and $t_2$ with attributes $A$ and $B$
where for every $a \in A$ where $t_1[a] \not = NULL$, $a \in B$ and $t_1[a] = t_2[a]$ are valid. Under this notation,
$t_2$ carries the entire inforation of $t_1$, but potentially contains more inforation for the $NULL$ entries of $t_1$.

For pIND discovery we now consider a \textit{subset} interpretation. Given two relational instances $r_1$ and $r_2$ with attribute sets
$X \in r_1$ and $Y \in r_2$ where $|X| = |Y|$. We define $r_1[X] \subseteq_\rho r_2[Y]$ as valid if there are at least $\lfloor |r_1[X]| \cdot \rho \rfloor$
tuples in $r_1[X]$ for which a \textit{more informative tuple} exists in $r_2[Y]$.

\subsection*{The \textit{foreign} interpretation}
A slight variation of the \textit{subset} interpretation. The difference being the added constraint that the referenced side is not allowed to
carry \textit{NULL} values. The motivation for this interpretation is the detection of foreign keys.
% TODO: Write how this only is a difference in the unary discovery, since the n-ary generation will automatically catch this constraint

\subsection*{The \textit{equality} interpretation}
This interpretation could be considered a somewhat optimistic version of the \textit{unknown} interpretation in a sense that we consider all $NULL$ entries as being equal. The result is equal to the subset approach, if we consider $NULL$ as just another explicit value. Logically this interpretation should be chosen if $NULL$ implicitly defines an value which can not be expressed explicitly (e.g. infinity in a numeric column), but still holds meaning and is expected in the referenced for a valid pIND.

\subsection*{The \textit{inequality} interpretation}
The pessimistic version of the \textit{unknown} interpretation. Here we consider every $NULL$ entry as an distinct value. Therefor the comparison between two entries $x$ and $y$ where one or both are $NULL$ always yields false. We are in a setting where $NULL$ entires are considered not yet filled and we choose a worst case scenario where the replacement values are all distinct form each other and the relation itself. In a non-partial setting this would restrict an attribute with at least one $NULL$ entry from forming any INDs. Applying a partial setting, there is still a chance, that attributes with only a few $NULL$ entries are able to form pINDs.

This interpretation is equal to the certain world concept. No matter how the $NULL$ values would be set, the found pINDs will stay valid.

\subsection*{Possible World}
In a possible world, all pINDs that could become valid under some $NULL$ replacement, are considered valid. This creates a whole new complexity layer when trying to choose the optimal $NULL$ replacements. If a referenced attribute has $n$ $NULL$ entries, we could avoid $n$ violations. If we do not care about deduplicates, this is easy to implementation, since we can increase the possible violations of a candidate by the number of $NULL$ in the referenced attribute. Should be care about deduplicates, this situation becomes more difficult. The optimal way of replacing $NULL$ would be to pick the $n$ values in the dependant attribute with the most occurrences which do not appear in the referenced. That way we avoid the highest number of violations. For this to work, we would need to keep track of the top $n$ highest occurring values, which create a violation. Since $n$ might be very big, we might not be able to store a priority queue or something similar in main memory, and would there for need to store these values on disk.

The described problems are certainly solvable, but are out of scope for the thesis. For this reason, the algorithms do not offer pIND discovery in a possible world setting.

\section{SPIND Overview}

We will now discuss our proposed algorithm $SPIND$, which stands for \textbf{S}calable \textbf{P}artial \textbf{IN}clusion \textbf{D}ependency discovery. Further the german word $Spind$ is a special kind of closet and often multiple "Spinds" are placed next to each other. This is a metaphor to the algorithms procedure. Every input relation will be transformed to a "Spind" of sorted values with connected attributes (attribute combinations), which is surely bigger than a $BINDER$ bucket, but there are far fewer "Spinds" than $BINDER$ would create buckets.

\noindent \\ \textit{SPIND} solves multiple problems which no other algorithm solves. The architecture does not need to monitor main memory consumption or estimate the size of some objects. \textit{BINDER} uses heuristics to estimate if the content of a file can be loaded without overflowing the available main memory. In some rare cases this heuristic can fail leading to an execution failure. Another issue is the creation of many files. On datasets with a lot of pINDs algorithms like \textit{BINDER} or \textit{SPIDER} will create one file per attribute (combination) regardless of the dataset size. This can lead to millions of files being created, which total a few megabytes. On top of the problems that \textit{SPIND} solves, it also outperforms \textit{BINDER} and \textit{SPIDER} computationally by a factor of up to multiple magnitudes.

\begin{figure}[h]
    \centering
    \includegraphics[width=0.49\textwidth]{figures/SPIND.pdf}
    \caption{Conceptual overview of \textit{SPIND}}
    \label{fig:spind}
\end{figure}

\subsection{Chunking the input relations}
Modern CPUs feature multiple cores capable of executing tasks concurrently. Previous studies have largely overlooked the potential of multi-threading. In contrast, $SPIND$ will leverage multi-threading extensively to maximize hardware utilization. Achieving this objective necessitates the identification of independent tasks capable of being processed in parallel.

\noindent \\ The execution starts by fetching some very basic information about the input relations. For each relational instance (input table) $SPIND$ will extract the header column (if present) and store the number of unary attributes (columns) each table has. Using a constant $CHUNK\_SIZE$ which is set by the user, we spilt each relation into somewhat equal parts. Each chunk will consist of at most $\lfloor \frac{CHUNK\_SIZE}{\# cloumns \: in \: relation} \rfloor$ rows. The complexity of processing a relation directly correlates to the number of total values in that relation. While this may be an oversimplification since there are many more factors, like the distribution of duplicate values, the raw size is easy to modify and a heuristic that can be applied without any specific dataset knowledge. Each of the resulting chunks is associated with exactly one relation and carries a subset of that relations rows.

\noindent \\ Chunking is done exactly once at the very start and is not repeated for n-ary layers. We reuse the same chunks in every layer of the n-ary pIND discovery.
% TODO Link section discussing chunking effectiveness.

\noindent \\ Since $SPIND$ almost always operates on the relation layer, a hash based partitioning, similar to $BINDER$, is not feasible. For the validation, we need to descend to the attribute layer and there we need to know which attributes (attribute combinations) share at least one value. A partitioning would therefore be required to split the dataset into $n$ groups $G_1, \dots, G_n$ such that for every tuple of values $t_i$ generate by some row, it holds that if $t_i \in G_j$ than 
\begin{itemize}
    \item[1)] $\forall \: t_k \in G \setminus G_j : t_i \cap t_k = \emptyset$
    \item[2)] and $t_i \cap t_k \not = \emptyset \: \forall t_k \in G_j$.
\end{itemize}
To find such groups we would need knowledge of all values in a relation. Even if we expand the groups row by row and merge when necessary, solving this grouping problem would create a lot of new complexity. This leads to the decision to use a plain horizontal splitting approach.

\noindent \\ At the end of the chunking phase we have a set of chucks for each relation, that all carry at most \textit{CHUNK\_SIZE} values. Larger relations have been split into more chunks while smaller relations are probably contained within a single chunk.

\subsection{Probabilistic Filtering}
Assume we would know the set $R$ of all values, which only occur in referred attributes. Such a set could be used to filter the values that would need to be sorted, merged and validated. How the filter is filled and where exactly it is used will be discussed in the sorting and validation sections. This section will explain the reason why such a filter can be used in the first way and which benefits it brings.

\noindent \\ Mathematically we require the set $R$ to hold the property $\forall \: ref \in candidates : ref \cap R = \emptyset$. Trivially, the empty set would be valid for $R$. Our goal is to find an $R$ which best approximates the union of all values present in referenced attributes (attribute combinations) without the values present in any dependant attributes.

\noindent \\ Since we may not be able to actually keep all of the values in main memory, we need a different approach of filtering the values. An important observation being, that we need to be certain, that all values which are present in at least one dependant attribute need to make it through the filter. If values which are only in dependant (or irrelevant) attributes make it through the filter, that creates more computational complexity, but will not create any incorrect results.

\noindent \\ These constraints can be full filled by building a probabilistic filter in which all values that appear in an dependant attribute at least once are stored. Such a filter can than be queried using a value to check if that value was added. If the value was added previously it will always return true. If the value was not added it may create a false positive at a rate of typically less than 5\%. There are multiple options for such filters.

\noindent \\ Research regarding probabilistic data structures focuses on two primary metrics govern research evaluation: the computational efficiency of insertion and query operations, and the required number of bits for storing individual values within the structure\cite{fan2014cuckoo}. Bloom filters are initialized with an anticipated number of elements and a target false positive rate. However, if additional values are incorporated beyond the initial expectation, Bloom filters may become oversaturated, leading to an approaching false positive rate of 1 \cite{tarkoma2011theory}. Bloom filters use about 44\% ($log_2(e)$) more space per key than the theoretical lower bound. There are probabilistic data structures which are closer to the theoretical lower bound, but these structures impose more complex strategies which are out of scope of this thesis \cite{fan2014cuckoo}.

\noindent \\ The size of the Bloom Filter is set to 100 million expected values with a false positive rate of 1\% by default. With these settings the filter will consume TODO bytes of main memory. If the available memory is very limited, the filter should be initialized with a smaller size or higher false positive rate. Table \ref{tab:filter} shows how many values are inserted into the bloom filter during u-nary pIND discovery for each dataset. In section \ref{sec:spind_val} we will discuss in detail how the values are inserted into the filter.

\begin{table}
    \begin{tabular}{c|c}
     dataset & values in filter\\ 
     \hline\hline
     data.gov & - \\ 
     \hline
     musicbrainz & - \\
    \end{tabular}
    \caption{The number of items that are inserted into the bloom filter for each dataset.} \label{tab:filter}
\end{table}

\subsubsection{Procedual Refining.} After each layer the set of values which are present in dependant attributes will be a subset or equal to the set of values of any of the preciding layers. If we construct the probabilistic filter after unary pINDs have been validated, we might can only masked out values based on the unary pINDs. At fours layer, the set of values that make up dependant attribute combinations might have shrunk significantly, but the filter could not accomondate for that. We could overcome this issue by reconstructing the filter during the validation of every layer. While this creates more computational effored it holds the potential of saving read and write operations during the sort and merge phases. To rebuild the filter in n-ary layer we use the inverse transformation (see Algorithm \ref{alg:string_transformation}) to extract the strings contained in the serialized value and compute the hashes. These steps are executed in the parallized section of the validation, while the insertion into the filter needs to be synchronized between threads.

\subsubsection{Filter evaluation.} In Figure TODO we find the execution times of all datasets (expect the most complex ones) using the bloom filter by constructing it once during u-nary discovery (\textit{Once}), using and refining the filter on every level (\textit{Refining}) and not using a probabilistic filter at all (\textit{None}). The y-Axis is relative to the longest execution time for each dataset and the displayed times are averaged over three executions for each mode. We find that using a filter that was build once or a refinded filter results in very similar execution times which are usually a bit faster than not using a filter. Since a refinded filter can save some read and write operations we will chose this version to put a smaller workload on the disk. Notice that the total execution time of \textit{Cars} is under one second for n-ary discovery, which results in the time it takes to construct the file to become quite significant. We will not give much weight to this observation since the dataset is so quick to solve anyway.


\subsection{Sorting Chunks}
% IDEA: spill based on occurrences to that point.
Our objective is to arrange each chunk in sorted order, a prerequisite for subsequent merging and validation procedures. Consistently, each chunk undergoes identical processing steps, executed concurrently.

We employ a \textit{CSVReader}\footnote{Provided by the open source library "Opencsv" \url{https://opencsv.sourceforge.net/}} for each chunk, specifying the relevant attribute combinations. Line by line processing of the chunk occurs, during which we update all associated attributes to their new value. In Section \ref{subsec:n-ary-stings} we discuss exactly how the values of n-ary attributes are assigned. For now we will take the existence of a proper string value for both unary and n-ary attributes as a given. To manage these values and their associations, we employ a hash based mapping structure where values are mapped to their respective attributes along with the frequency of occurrences, thus forming a nested map. To prevent main memory overflows, we constrain the size of this map using a constant denoted as $SORT\_SIZE$. If the map size reaches the specified threshold (which is divided by the number of threads) or when sorting is complete, the contained data is serialized to disk.

This storage process will first sort the key set of the outer map. Afterwards we iterate over the sorted entries and write two lines for each of them. The first line contains the string value, which is the entries key. Followed by the the attributes which formed the value combined with the occurrences. Thereby the second line persists the value of the entry. Figure \ref{fig:sorting} illustrates this workflow while sorting for unary pIND discovery. The Attributes are $1:$\textit{id}, $2:$\textit{postCode}, $3:$\textit{landline}. For every spilled file, we retain the path of the sorted (sub) chunk, ultimately returning a list of these paths.

A minor optimization conduct here is spilling values based on the sum of occurrences in the connected attributes. We know that in a single chunk, there are at most \textit{CHUNK\_SIZE} unique values. In n-ary layers there might be relations where the number candidates of than relation is greater than the number of columns. Therefor we can calculate that the maximal possible number of distinct values generated from a chunk is $N = \underbrace{\lfloor \frac{CHUNK\_SIZE}{\#cloumns} \rfloor}_{= \#rows} \cdot \#candidates$, since every row can produce at most one unique value for each candidate. We can now say that $\frac{N}{SORTED\_SIZE}$ is the largest possible minimum of occurrences over all present values. The proposed heuristic suggests to keep all values with more than $\frac{\#candidates \cdot CHUNK\_SIZE}{SORTED\_SIZE}$ occurrences which is greater than $\frac{N}{SORTED\_SIZE}$ in main memory and only spill those values which where not seen often. The idea being, that values which have occurred often will be more likely to occur often. Spilling these last can save complexity during merging since less candidate occurrences need to be deserialized.

\begin{figure}[h]
    \centering
    \includegraphics[width=0.45\textwidth]{figures/Sorting.pdf}
    \caption{Simplified illustration of the sorting process.}
    \label{fig:sorting}
\end{figure}

\subsection{Merging (spilled) Chunks}
We now have a bunch of files which are all sorted by themselves. The next step is to merge all files which originated from the same relation. Again, this is computed in parallel. To avoid too many files being opened at the same time the constant \textit{MERGE\_SIZE} can be used to limit the number of files which are being sorted per thread. Due to multithreading the actual number of simultaneously opened files is $\#threads \cdot (MERGE\_SIZE + 1)$. The plus one originates since we always need to open the resulting output file of a merge as well. We attach a buffered reader to each file and use a priority queue to perform a k-way merge \ref{taniar2008high}. At the end of the merge phase, we have constructed a single sorted file for each relation.

\subsection{Validation}\label{sec:spind_val}
The validation process expects a sorted file per relation. The validation process works analog to the validation performed by \textit{SPIDER}. Given the sorted files we check which attributes (attribute combinations) share a value and move the readers forward bit by bit in a way that only those readers are updated which share the same smallest value. \textit{SPIDER} uses a priority queue which stores the head value of each reader. Such a data structure enables us to calculate the smallest of the head values in $log(M)$ time, $M$ being the number of attributes (attribute combinations). Assume the complexity of pruning the connected attributes to some value is $O(k)$. Validating all values $N$ would cost $O(k \cdot N \cdot log(M))$. Pruning is a synchronous task, meaning the active candidates need to be synchronized between every prune. We will therefor try to create the attribute groups in parallel and conduct pruning in the main thread.

\noindent \\ We introduce the last variable \textit{VALIDATION\_SIZE}. It is a fixed value which states how many values are loaded at the same time for each relation. Deserialization is computed in parallel for all readers and we calculate the smallest largest value for each relation queue after they have been refilled. This largest smallest queue value is calculate at most $\lceil\frac{N}{\textit{VALIDATION\_SIZE}}\rceil$ times if all $N$ values distributed over the different relations where distinct. The value groups are then also build in parallel using a stream architecture to efficiently include the synchronized pruning. The resulting complexity is  $O(k \cdot N + \lceil M \cdot \frac{N}{\textit{VALIDATION\_SIZE}}\rceil)$. Should $M$, the number of relations, be very large, the logarithm would yield a better theoretical complexity eventually. But since $M$ can be expected to be much smaller than the \textit{VALIDATION\_SIZE} and the proposed version operates mostly in parallel, we find that this concept of loading groups in parallel decreases the compute time by a factor of up to five and especially helps when processing large datasets.



\section{Hyperparameter Optimization}\label{subsec:hyperparameters}

There are five hyperparameters that affect the performance of \textit{SPIND}. These include the size of the initially generated chunks (\textit{CHUNK\_SIZE}), the maximum number of values retained in main memory during the sorting phase (\textit{SORT\_SIZE}), the maximum number of files merged simultaneously (\textit{MERGE\_SIZE}), the total queue size across all relations during candidate validation (\textit{VALIDATION\_SIZE}), and the level of parallelization (\textit{PARALLELISM}) in all phases of the execution. Using Bayesian optimization \cite{shahriari2015taking}, we iteratively identify parameter configurations that minimize \textit{SPIND}'s execution time across datasets. We set the minimum chunk size to 10,000 to prevent the creation of an excessive number of files, with a maximum limit of 100 million. During sorting, the maps are constrained between 100,000 and 50 million to strike a balance between the number of files generated and staying within the main memory limit of 20 GB. The number of files merged at the same time is restricted to a minimum of two and a maximum of 2,000 to avoid overloading the file system. The buffers for the relations during validation are capped at 1 million. Lastly, the level of parallelization is capped by the number of virtual threads available on the executing machine (twelve).

The experiments are carried out using the datasets \textit{EU}, \textit{US}, and \textit{TPC-H 1}, \textit{Population}, \textit{UniPort}. These datasets have varying structures, and optimizing the parameters for all of them at the same time will assist us in identifying robust configurations that do not overfit any specific dataset structure. 

The first observation is that the degree of \textit{PARALLELISM} has a clear negative correlation with the execution time. We find that the fastest executions use at least 9 threads, regardless of how the other four parameters are set. The subsequent observation indicates that \textit{CHUNK\_SIZE} is the second most significant hyperparameter. Although smaller chunks are directly related to the total number of files created and the associated I/O overhead, an optimal chunk size of five to seven million was identified, which is also stable for \textit{Musicbrainz}, \textit{TPC-H 10}, and \textit{UniProt}. Smaller chunk sizes allow for more parallel task processing, which demonstrates mitigation of I/O operations at a certain point.

The remaining hyperparameters are far less influential and do not show a clear pattern on their own. We establish a stable configuration with a chunk size of 6 million. Sort maps are limited to 25 million nested entries, files merged simultaneously to 1.200, validation buffers to 250,000, and parallelization to 12. The configuration will be used for the execution times shown in Table \ref{tab:runtimes}.

\begin{figure*}[!t]
    \centering
        \includegraphics[width=.98\textwidth]{figures/sensitivity.pdf}
    \caption{Sensitivity around the hyperparameters obtained through Bayesian optimization}
    \label{fig:hyperparameters}
\end{figure*}

A sensitivity analysis indicated that the five parameters maintain stability close to the optimal values identified by Bayesian optimization. Figure \ref{fig:hyperparameters} presents a subplot for each hyperparameter. The x-axis represents the hyperparameter values, while the y-axis, shared between plots, depicts the relative runtime compared to the run with the Bayesian optimization settings. In the \textit{CHUNK\_SIZE} plot, it is evident that for some datasets, an alternative configuration might perform better. With more knowledge of the data set, deviating from our recommended hyperparameter settings could be beneficial.




\chapter{Datasets}
To understand the performance of the proposed algorithms it is crucial to perform testing on a variety of data sets. For this purpose we will gather some real word data sets. Further we will create synthetic data sets that aim on edge cases to see if the performance is strongly dependent on structural assumptions.

\section{Real World Data Sets}
There are many sources for csv or tsv files online. I have decided to gather data from the US Government\footnote{\href{https://data.gov}{data.gov}}, the European Union\footnote{\href{https://data.europe.eu}{data.europe.eu}}, Kaggle\footnote{\href{https://kaggle.com}{Kaggle.com}}, Musicbrainz\footnote{\href{https://musicbrainz.org/}{musicbrainz.org}}, and Eurostat\footnote{\href{https://musicbrainz.org/}{ec.europa.eu}}. Further data set sources may be added. Related research papers sometimes discuss the origin of the used data, do not discuss the structure of the data they use \cite{papenbrock2017data,bauckmann2006efficiently, dursch2019inclusion, rostin2009machine}. In order to understand the resulting algorithm performance, we believe it is crucial to examine the data which is tested against. In this section we will discuss the data used and later try to understand why an algorithms performance may vary over different test sets.

\section{Synthetic Data Sets}
To evaluate the proposed algorithms under detailed aspects, we will generate synthetic data sets. The strategies and claims are based on \cite{jordon2022synthetic} synthetic data can be defined as \textit{data that has been generated using a purpose-built mathematical model or algorithm, with the aim of solving a (set of) data science task(s).} While we will not try to train a model with the synthetic data, it is still of great use for us, since we have absolute knowledge about the underlying structures. The decision is based on the fact that there is a lot of real word data available, since open data is a growing market which expected to grow even further \cite{EUopenData}. Synthetic data on the other hand enables us to evaluate the algorithm performances on edge cases, which we may not be able to find in the selection of real world data sets. \\

\noindent To test certain edge cases of the proposed algorithms, we will construct various edge case data sets. The \textit{SameSame} dataset consists of 32 attributes and 250.000 records. Each attribute carries the numbers 1 to 250.000 in the natural order. This means every attribute is a (partial) inclusion dependency of every other attribute. The same obviously also holds for combinations of columns. We will now calculate the expected number if (p)INDs in each layer. Since all candidates are perfect matches, the chosen threshold $\rho$ will not influence the number of pINDs. Table % TODO add ref
shows the number of candidates/pINDs for the \textit{SameSame} dateset.
% TODO calculate INDs
The edge case to test here is, how well the algorithm can understand equality relations and prune the candidate space. While this may seem like an unlikely edge case we will also investigate how often this happens in real world data sets. \\
% write about real world structures

\noindent Another source of synthetic data will be the TPC (Transaction Processing Performance Council) Benchmarks \footnote{\href{https://www.tpc.org/}{tpc.org}}. The TPC Benchmarks are a set of standardized and vendor-neutral performance benchmarks used to evaluate the processing and database capabilities of different systems. These benchmarks are designed to model various types of workloads. The TPC-E benchmark, for example, models a brokerage firm with customers who generate transactions related to trades, account inquiries, and market research, while the TPC-C benchmark is intended to model a medium complexity online transaction processing workload, patterned after an order-entry system with skewed access within individual data types/relations. Using scaling factors, a user can define the size of the synthetic database themselves. This enables us to examine the algorithm performances in a very controllable setting.



% This document should discuss passed approaches focusing on pros and cons

\section{Related Work}\label{sec:rel_work}

Inclusion dependencies (INDs) are a highly influential concept in both database research and practice, with a wide range of contributions and applications. The introduction already provided some insight into their diverse application areas of INDs. In this section, we will focus on the key achievements related to the implication problem of INDs. We will go over different algorithms and discuss their unique features. \\

In 1981 INDs started as a general notation of referential integrity, which was already a well established concept back then \cite{date1981referential}. Casanova et al. presented a paper on the inference rules of INDs\cite{casanova1982inclusion}. Three axioms where introduced: \textit{reflexivity}, \textit{transitivity} and \textit{projection and permutation}. The application of these rules to partial INDs will later be discussed in detail (Section \ref{theo:pInd}). The paper further proofed that the discovery of INDs is PSPACE-complete if there is no limit on the size of inclusions. Publications typically fall into three groups of algorithms, foreign key discovery algorithms, unary IND discovery and n-nary IND discovery \cite{papenbrock2017data}. \\

In 1995 Bell and Brockhausen \cite{bell1995discovery} propose a graph-based approach to represent the relationships between attributes, allowing for a more efficient exploration of the search space. The algorithm is initiated with a directed graph, wherein all possible edges, which could not be pruned by statistical measures, are included. A directed edge in the graph represents an inclusion dependency, which is read as the edge from $A_i$ to $A_j$ ($A_i \rightarrow A_j$) represents the IND $A_i \subseteq A_j$. It then proceeds to remove those edges that failed the IND check. To determine the validity of an edge, the algorithm checks for transitivity, which enables it to answer whether a dependency could exist between two variables based on their relationships with a third variable, which was tested previously. If it is ascertained that a dependency is impossible, the algorithm skips the test and directly removes the given edge, thereby reducing the overall computational cost. The approach presented by Bell and Brockhausen for unary IND discovery has both reusable aspects and downsides. The algorithm's candidate generation technique, which uses data statistics such as data types and min-max values, can be reused in other discovery algorithms. This preprocessing step reduces the number of candidates that need to be validated and further reduce the storage overhead needed to store candidates. However, the validation technique used in the algorithm, which relies on SQL join-statements and requires accessing the data on disk, is infeasible for larger candidate sets. This limits the scalability of the approach and makes it less practical for large-scale data sets. Additionally, the need to store the data in a database and access it on disk for validation can add to the computational cost and time required for the discovery process.\\

The \textit{SPIDER} algorithm is a disk-backed, all-column sort-merge join with early termination used for the discovery of inclusion dependencies \cite{bauckmann2006efficiently}. It sorts the values of each attribute, removes duplicate values, and writes the results to disk in the first phase. In the second phase, it performs the actual inclusion dependency discovery by using a pointer for each file and validating all candidates at the same time. A major advantage is, that in this setting every value only needs to be read a single time from disk, which greatly reduces the I/O bottleneck. The Spider algorithm has been the subject of experimental evaluation and is considered one of the efficient techniques for unary IND discovery. Still, there are drawback if the data set is too big to be sorted in main memory or if the number of simultaneously open files allowed by the OS system is reached \cite{papenbrock2015divide}. \\

In 2009 Bauckmann et al. also proposed a partialized version of $SPIDER$ in a section of the same paper. The authors found that there where surprisingly many partial inclusions (under a 5\% threshold) in their test data sets. To find partial inclusion dependencies, they first count how many distinct violations are present and in a second step consider the amount of duplicates for not included values. This means their algorithm does not immediately stop once a single validation has been found but only after an added counter surpasses a given threshold. The paper is not particular clear on how the number of duplicates is stored/retrieved and additionally does not analyse the computational effect of these changes and with the original source code being lost, this a approach can only be verified using a best guess approach.\\

The \textit{SAWFISH} algorithm \cite{kaminsky2023discovering}, published in 2023, is designed for identifying similarity inclusion dependencies (sINDs) within datasets, introducing a novel perspective on inclusion dependencies (INDs). While traditional INDs assume error-free data, \textit{SAWFISH} incorporates a similarity measure to accommodate minor errors like typos. Given a similarity threshold $\omega$ a sIND is valid if and only if for all tuples in the left hand side there exists a tuple in the right hand side which has at least a similarity of $\omega$ under a set similarity measure. The authors used the edit distance as well as the normalized edit distance. Through preprocessing, metadata generation, and a sliding-window approach, \textit{SAWFISH} successfully identifies and validates sIND candidates using an inverted index, providing a valuable tool for database applications despite dirty data challenges. \\



\section{Core Structural Elements}

Nulla placerat feugiat augue, id blandit urna pretium nec. Nulla velit sem, tempor vel mauris ut, porta commodo quam. Donec lectus erat, sodales eu mauris eu, fringilla vestibulum nisl. Morbi viverra tellus id lorem faucibus cursus. Quisque et orci in est faucibus semper vel a turpis. Vivamus posuere sed ligula et. 

\subsection{Figures}

Aliquam justo ante, pretium vel mollis sed, consectetur accumsan nibh. Nulla sit amet sollicitudin est. Etiam ullamcorper diam a sapien lacinia faucibus. Duis vulputate, nisl nec tincidunt volutpat, erat orci eleifend diam, eget semper risus est eget nisl. Donec non odio id neque pharetra ultrices sit amet id purus. Nulla non dictum tellus, id ullamcorper libero. Curabitur vitae nulla dapibus, ornare dolor in, efficitur enim. Cras fermentum facilisis elit vitae egestas. Nam vulputate est non tellus efficitur pharetra. Vestibulum ligula est, varius in suscipit vel, porttitor id massa. Nulla placerat feugiat augue, id blandit urna pretium nec. Nulla velit sem, tempor vel mauris ut, porta commodo quam \autoref{fig:duck}.

\begin{figure}
  \centering
  \includegraphics[width=\linewidth]{figures/duck}
  \caption{An illustration of a Mallard Duck. Picture from Mabel Osgood Wright, \textit{Birdcraft}, published 1897.}
  \label{fig:duck}
\end{figure}

\begin{table*}[t]
  \caption{A double column table.}
  \label{tab:commands}
  \begin{tabular}{ccl}
    \toprule
    A Wide Command Column & A Random Number & Comments\\
    \midrule
    \verb|\tabular| & 100& The content of a table \\
    \verb|\table|  & 300 & For floating tables within a single column\\
    \verb|\table*| & 400 & For wider floating tables that span two columns\\
    \bottomrule
  \end{tabular}
\end{table*}

\subsection{Tables}

Curabitur vitae nulla dapibus, ornare dolor in, efficitur enim. Cras fermentum facilisis elit vitae egestas. Mauris porta, neque non rutrum efficitur, odio odio faucibus tortor, vitae imperdiet metus quam vitae eros. Proin porta dictum accumsan \autoref{tab:commands}.

Duis cursus maximus facilisis. Integer euismod, purus et condimentum suscipit, augue turpis euismod libero, ac porttitor tellus neque eu enim. Nam vulputate est non tellus efficitur pharetra. Aenean molestie tristique venenatis. Nam congue pulvinar vehicula. Duis lacinia mollis purus, ac aliquet arcu dignissim ac \autoref{tab:freq}. 

\begin{table}[hb]% h asks to places the floating element [h]ere.
  \caption{Frequency of Special Characters}
  \label{tab:freq}
  \begin{tabular}{ccl}
    \toprule
    Non-English or Math & Frequency & Comments\\
    \midrule
    \O & 1 in 1000& For Swedish names\\
    $\pi$ & 1 in 5 & Common in math\\
    \$ & 4 in 5 & Used in business\\
    $\Psi^2_1$ & 1 in 40\,000 & Unexplained usage\\
  \bottomrule
\end{tabular}
\end{table}

Nulla sit amet enim tortor. Ut non felis lectus. Aenean quis felis faucibus, efficitur magna vitae. Curabitur ut mauris vel augue tempor suscipit eget eget lacus. Sed pulvinar lobortis dictum. Aliquam dapibus a velit.

\subsection{Listings and Styles}

Aenean malesuada fringilla felis, vel hendrerit enim feugiat et. Proin dictum ante nec tortor bibendum viverra. Curabitur non nibh ut mauris egestas ultrices consequat non odio.

\begin{itemize}
\item Duis lacinia mollis purus, ac aliquet arcu dignissim ac. Vivamus accumsan sollicitudin dui, sed porta sem consequat.
\item Curabitur ut mauris vel augue tempor suscipit eget eget lacus. Sed pulvinar lobortis dictum. Aliquam dapibus a velit.
\item Curabitur vitae nulla dapibus, ornare dolor in, efficitur enim.
\end{itemize}

Ut sagittis, massa nec rhoncus dignissim, urna ipsum vestibulum odio, ac dapibus massa lorem a dui. Nulla sit amet enim tortor. Ut non felis lectus. Aenean quis felis faucibus, efficitur magna vitae. 

\begin{enumerate}
\item Duis lacinia mollis purus, ac aliquet arcu dignissim ac. Vivamus accumsan sollicitudin dui, sed porta sem consequat.
\item Curabitur ut mauris vel augue tempor suscipit eget eget lacus. Sed pulvinar lobortis dictum. Aliquam dapibus a velit.
\item Curabitur vitae nulla dapibus, ornare dolor in, efficitur enim.
\end{enumerate}

Cras fermentum facilisis elit vitae egestas. Mauris porta, neque non rutrum efficitur, odio odio faucibus tortor, vitae imperdiet metus quam vitae eros. Proin porta dictum accumsan. Aliquam dapibus a velit. Curabitur vitae nulla dapibus, ornare dolor in, efficitur enim. Ut maximus mi id arcu ultricies feugiat. Phasellus facilisis purus ac ipsum varius bibendum.

\subsection{Math and Equations}

Curabitur vitae nulla dapibus, ornare dolor in, efficitur enim. Cras fermentum facilisis elit vitae egestas. Nam vulputate est non tellus efficitur pharetra. Vestibulum ligula est, varius in suscipit vel, porttitor id massa. Cras facilisis suscipit orci, ac tincidunt erat.
\begin{equation}
  \lim_{n\rightarrow \infty}x=0
\end{equation}

Sed pulvinar lobortis dictum. Aliquam dapibus a velit porttitor ultrices. Ut maximus mi id arcu ultricies feugiat. Phasellus facilisis purus ac ipsum varius bibendum. Aenean a quam at massa efficitur tincidunt facilisis sit amet felis. 
\begin{displaymath}
  \sum_{i=0}^{\infty} x + 1
\end{displaymath}

Suspendisse molestie ultricies tincidunt. Praesent metus ex, tempus quis gravida nec, consequat id arcu. Donec maximus fermentum nulla quis maximus.
\begin{equation}
  \sum_{i=0}^{\infty}x_i=\int_{0}^{\pi+2} f
\end{equation}

Curabitur vitae nulla dapibus, ornare dolor in, efficitur enim. Cras fermentum facilisis elit vitae egestas. Nam vulputate est non tellus efficitur pharetra. Vestibulum ligula est, varius in suscipit vel, porttitor id massa. Cras facilisis suscipit orci, ac tincidunt erat.

%%% End of paper content %%%

%\clearpage

\bibliographystyle{ACM-Reference-Format}
\bibliography{sample}

\end{document}
\endinput
