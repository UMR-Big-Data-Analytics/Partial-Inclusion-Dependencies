\subsection{Datasets}\label{subsec:datasets}
There are many sources for csv or tsv files online that span a range of domains. We aim to test the proposed algorithm on a collection of data sets that vary in size and domain. Table \ref{tab:datasets} contains information on the characteristics of each data set. The runtimes stated are calcualted using the \textit{subset} \textit{NULL} interpretation in combination with \textit{duplicateAware} handling (see Section \ref{sec:null_subset}, Section \ref{sec:foundations}).

\begin{table*}[t]
    \centering
    \begin{tabular}{llrrrrrr}
        \hline
        \textbf{Name} & \textbf{Domain} & \textbf{Size on Disk} & \textbf{Relations} & \textbf{Attributes} & \textbf{Unaries} & \textbf{N-aries} & $\textbf{n}_\textbf{max}$ \\
        \hline
        Cars & Retail & 6.1 MB & 13 & 117 & 281 & 91 & 4 \\
        ACNH & Video-Games & 3.5 MB & 30 & 630 & 8,686 & 20,908,814 & 12 \\
        T2D & Benchmark & 4 MB & 669 & 3125 & 362,604 & 9,301,847 & 8 \\
        WebTables & Various & 5.3 MB & 5,000 & 18,663 & 19,924,741 & - & 1$^\dag$ \\
        US & Governmental & 1.2 GB & 16 & 255 & 753 & 215,308 & 7 \\
        EU & Governmental & 1.8 GB & 37 & 624 & 18,752 & 54,634 & 6 \\
        Population & Demographics & 1.7 GB & 1 & 109 & 236 & 1 & 2 \\
        Musicbrainz & Entertainment & 16.7 GB & 171 & 301 & 1,843 & 2,739,733 & 4$\dag$ \\
        UniProt & Biology & 633 MB & 263 & 2367 & 420,412 & 1,174,863 & 5 \\
        Tesma & Synthetic & 80 MB & 2 & 12 & 4 & 1 & 2 \\
        TPC-H 1 & Synthetic & 1 GB & 7 & 61 & 96 & 8 & 2 \\
        TPC-H 10 & Synthetic & 10.5 GB & 7 & 61 & 97 & 11 & 3 \\
        \hline
    \end{tabular}
    \caption{Datasets and their characteristics. Max n-ary layers marked with $^\dag$ are user defined limits.}
    \label{tab:datasets}
\end{table*}

\textbf{Cars} is a small dataset regarding used cars and their retail prices. The different relations each focus on a single brand (e.g. Audi, Ford or Skoda). Each row of a relation contains information on a specific model of that brand. The smallest dataset by required disk size, \textbf{ACNH}, contains data form the video game Animal Crossing: New Horizons\footnote{\url{https://animalcrossing.nintendo.com/new-horizons/} (Last Access: 30/06/2024)}. This dataset contains information on all items, recipes, and achievements in the game. It was selected since it contains a vast amount of deep (p)INDs a poses the challenge of efficient candidate handling. \textbf{T2D} is a gold standard for matching web tables to DBpedia\footnote{\url{https://www.dbpedia.org/} (Last Access: 30/06/2024)}. We use a subset of the 779 provided web table that span various domains. The subset includes all tables with at least five rows. Similar to \textbf{T2D}, \textbf{WebTables} offers an even larger collection. WebDataCommons published a random sample of their Web Table Corpus\footnote{\url{https://webdatacommons.org/webtables/2015/downloadInstructions.html} (Last Access: 30/06/2024)}. We use this sample to evaluate the capabilities when a massive amount of relations and attributes are present. We will not try to compute the n-ary pINDs for this data set. The \textbf{US} government and the European union (\textbf{EU}) both publicly share regional, national, and international data. The two dataset that are build on these sources represent the governmental domain. \textbf{Population} is a wide table containing world wide demographic data. It includes the population sizes for different ages in various countries from 1950, including predictions, up to 2025. \textbf{Musicbrainz} is a repository of music knowledge. The dataset in relational form has a large number of attributes, relations and rows. It is the most computational complex of the chosen datasets. The biological domain is covered by \textbf{UniProt}. This data set contains vertebrate genomes acquired from Ensembl \footnote{\url{https://www.ensembl.org/index.html} (Last Access: 30/06/2024)} in the UniProt\footnote{\url{https://www.uniprot.org/} (Last Access: 30/06/2024)} standard. Lastly, we have the synthetic datasets \textbf{Tesma} and \textbf{TPC-H}. \textbf{Tesma} is a database generation tool developed by the Hasso Plattner Institute. Using a configuration file with relational constraints and the desired size, it will create multiple csv files with the wanted structure. \textbf{TPC-H} is a benchmark for database performance. Using the generation tool, we have created a 1 GB version (\textbf{TPC-H 1}) and a 10 GB version (\textbf{TPC-H 10}). To enable reproducibility, the connected GitHub repository contains a detailed technical documentation on how each dataset was acquired\footnote{\url{https://github.com/Jakob-L-M/partial-inclusion-dependencies/tree/main/data} (Last Access: 30/06/2024)}.

\section{Hyperparameter Optimization}\label{subsec:hyperparameters}

There are five hyperparameters that affect the performance of \textit{SPIND}. These include the size of the initially generated chunks (\textit{CHUNK\_SIZE}), the maximum number of values retained in main memory during the sorting phase (\textit{SORT\_SIZE}), the maximum number of files merged simultaneously (\textit{MERGE\_SIZE}), the total queue size across all relations during candidate validation (\textit{VALIDATION\_SIZE}), and the level of parallelization (\textit{PARALLELISM}) in all phases of the execution. Using Bayesian optimization \cite{shahriari2015taking}, we iteratively identify parameter configurations that minimize \textit{SPIND}'s execution time across datasets. We set the minimum chunk size to 10,000 to prevent the creation of an excessive number of files, with a maximum limit of 100 million. During sorting, the maps are constrained between 100,000 and 50 million to strike a balance between the number of files generated and staying within the main memory limit of 20 GB. The number of files merged at the same time is restricted to a minimum of two and a maximum of 2,000 to avoid overloading the file system. The buffers for the relations during validation are capped at 1 million. Lastly, the level of parallelization is capped by the number of virtual threads available on the executing machine (twelve).

The experiments are carried out using the datasets \textit{EU}, \textit{US}, and \textit{TPC-H 1}, \textit{Population}, \textit{UniPort}. These datasets have varying structures, and optimizing the parameters for all of them at the same time will assist us in identifying robust configurations that do not overfit any specific dataset structure. 

The first observation is that the degree of \textit{PARALLELISM} has a clear negative correlation with the execution time. We find that the fastest executions use at least 9 threads, regardless of how the other four parameters are set. The subsequent observation indicates that \textit{CHUNK\_SIZE} is the second most significant hyperparameter. Although smaller chunks are directly related to the total number of files created and the associated I/O overhead, an optimal chunk size of five to seven million was identified, which is also stable for \textit{Musicbrainz}, \textit{TPC-H 10}, and \textit{UniProt}. Smaller chunk sizes allow for more parallel task processing, which demonstrates mitigation of I/O operations at a certain point.

The remaining hyperparameters are far less influential and do not show a clear pattern on their own. We establish a stable configuration with a chunk size of 6 million. Sort maps are limited to 25 million nested entries, files merged simultaneously to 1.200, validation buffers to 250,000, and parallelization to 12. The configuration will be used for the execution times shown in Table \ref{tab:runtimes}.

\begin{figure*}[!t]
    \centering
        \includegraphics[width=.98\textwidth]{figures/sensitivity.pdf}
    \caption{Sensitivity around the hyperparameters obtained through Bayesian optimization}
    \label{fig:hyperparameters}
\end{figure*}

A sensitivity analysis indicated that the five parameters maintain stability close to the optimal values identified by Bayesian optimization. Figure \ref{fig:hyperparameters} presents a subplot for each hyperparameter. The x-axis represents the hyperparameter values, while the y-axis, shared between plots, depicts the relative runtime compared to the run with the Bayesian optimization settings. In the \textit{CHUNK\_SIZE} plot, it is evident that for some datasets, an alternative configuration might perform better. With more knowledge of the data set, deviating from our recommended hyperparameter settings could be beneficial.




\subsection{Algorithm Runtimes} \label{subsec:runtime}
Table \ref{tab:runtimes} contains information on the total execution time of all data sets for the algorithms discussed for the discovery of unary and nary IND. All runs have been executed on the same machine, which is equipped with an AMD Ryzen 5 3600X, 32GB 3200MHz DDR4 RAM and dual Samsung 950 EVO SSDs. Each algorithm was limited to a maximal consumption of 20 GB RAM. We perform IND discovery to understand the execution times of the partial variants (\textit{pSPIDER}, \textit{pBINDER}, \textit{SPIND}) in comparison to known references (\textit{SPIDER}, \textit{BINDER}).

The memory overflow issue observed with the \textit{WebTables} dataset arises from \textit{SPIDER} employing HashMaps for candidate storage, which demands significantly more memory compared to using single linked lists. When discovering n-ary INDs, \textit{BINDER} did not finish the \textit{ACNH} and \textit{T2D} datasets. For \textit{ACNH} \textit{BINDER} was not able to finish the fourth layer in four hours and for \textit{T2D} the sixth layer could not be completed. In the case of \textit{ACNH} we can conclude that \textit{BINDER} was unable to complete $\sim18\%$ of the total candidates. When estimating the total run time using the growth of the attribute space in the deeper layers, we would expect \textit{BINDER} to take at least 24 hours, but most likely much longer due to poorer candidate generation. Using the same approximation, \textit{T2D} would also take over 24 hours. In addition, variants create millions of files and manually call the JVM garbage collector constantly, which causes execution to be very impractical. \textit{BINDER} and \textit{pBINDER} where also unable to finish \textit{Musicbrainz} within 12h. Both executions did not surpass the third layer while the second layer was completed after 5h 13m and 5h 4m respectively. \textit{SPIND} finished all data sets faster than the \textit{BINDER} variants. We find that data sets showing complex and deep (p)INDs are solved much faster by \textit{SPIND} while the ones with shallow (p)INDs can also be solved using \textit{BINDER} or \textit{pBINDER}.

When discovering unary INDs, we find \textit{pSPIDER} to present faster execution times. Parallelization and more efficient structures allow the algorithm to beat the original version by a factor of up to 14x (\textit{UniProt}). \textit{BINDER} and \textit{pBINDER} are able to narrow this factor, especially for larger data sets, but are still unable to beat \textit{pSPIDER} in any data set. \textit{SPIND} is structurally very similar to \textit{pSPIDER} when discovering unary INDs. Contrary to \textit{SPIND}, merging and validation are not performed at the same time, which causes \textit{SPIND} to take slightly longer while still outperforming the \textit{BINDER} variants.


\begin{table}[]
    \centering
    \resizebox{.475\textwidth}{!}{
        \begin{tabular}{lrrrrrr}
        \hline
        \textbf{Unary} & \textbf{\footnotesize SPIDER} & 
        \textbf{\footnotesize pSPIDER} &
        \textbf{\footnotesize BINDER} & 
        \textbf{\footnotesize pBINDER} &  
        \textbf{\footnotesize SPIND} \\
        \hline
        Cars & 1.1s & 0.5s & 1.3s & 1.3s & \textbf{0.2s} \\
        ACNH & 1.3s & 0.6s & 2.3s & 1.9s & \textbf{0.1s} \\
        T2D & 4.6s & 1.5s & 9.3s & 9.4s & \textbf{1.2s} \\
        WebTables & \textit{OOM} & \textbf{15.6s} & 1m 46s & 1m 19s & 16.0s \\
        US & 1m 58s & \textbf{14.0s} & 49.4s & 52.7s & 21.7s \\
        EU & 2m 5s & \textbf{21.3s} & 49.8s & 51.0s & 30.4s \\
        Population & 5m 36s & \textbf{34.4s} & 2m 36s & 2m 44s & 1m 32s \\
        Musicbrainz & 43m 20s & \textbf{10m 58s} & 24m 2s & 24m 41s & 12m 4s \\
        UniProt & 1m 43s & \textbf{7.0s} & 45.8s & 45.3s & 12.4s \\
        Tesma & 8.9s & \textbf{1.8s} & 3.8s & 4.2s & 3.8s \\
        TPC-H 1 & 2m 14s & \textbf{24.9s} & 55.1s & 53.4s & 38.1s \\
        TPC-H 10 & 27m 29s & \textbf{4m 57s} & 9m 19s & 9m 15s & 6m 10s \\
        \hline
        \hline
        \textbf{N-ary} & \textbf{\footnotesize SPIDER} & 
        \textbf{\footnotesize pSPIDER} &
        \textbf{\footnotesize BINDER} & 
        \textbf{\footnotesize pBINDER} &  
        \textbf{\footnotesize SPIND} \\
        \hline
        Cars & - & - & 2.8s & 2.8s & \textbf{1.5s} \\
        ACNH & - & - & \textit{DNF} & \textit{DNF} & \textbf{20m 35s} \\
        T2D & - & - & \textit{DNF} & \textit{DNF} & \textbf{33.3s} \\
        US & - & - & 3h 26m & 25min 39s & \textbf{2m 13s} \\
        EU & - & - & 2m 44s & 2m 23s & \textbf{44.0s} \\
        Population & - & - & 2h 40m & 2h 42m & \textbf{1h 58m} \\
        Musicbrainz & - & - & \textit{DNF} & \textit{DNF} & \textbf{2h 7m} \\
        UniProt & - & - & 7m 20s & 4m 25s & \textbf{25.1s} \\
        Tesma & - & - & 9.5s & 9.3s & \textbf{6.1s} \\
        TPC-H 1 & - & - & 4m 51s & 4m 40s & \textbf{4m 11s} \\
        TPC-H 10 & - & - & 56m 38s & 57m 3s & \textbf{47m 3s} \\
        \hline
    \end{tabular}
    }
    \caption{Runtime evaluation over the different datasets in both unary (top) and n-ary (bottom) settings. \textit{OOM}: Out of Memory, \textit{DNF}: Did not finish}.
    \label{tab:runtimes}
\end{table}

