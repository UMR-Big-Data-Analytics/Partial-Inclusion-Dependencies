% In this document I want to talk about storage for relational databases SQL, csv files.
% Explain that I will develop in Java
Discovering inclusion dependencies is not a new problem. Over the years researchers have used a variety of technologies to improve performance. Starting with storage the two most common assumptions are either a relational database or structured storage files (e.g. CSV or TSV files). There are algorithms which rely on SQL for IND candidate validation \cite{bell1995discovery}.
Still, the techniques used in the algorithms are not dependent on the input format or language of implementation. This is why we discuss algorithms without setting constraints on the input format or validation strategy. The thesis will assume the input to always be static files. There have been efforts to collect all kinds of IND discovery algorithms and integrate them into a single framework \cite{dursch2019inclusion}. The Metanome platform \cite{Papenbrock:2015:DPM:2824032.2824086} provides that framework for the stated implementations and this thesis will use Metanome to execute tests on existing algorithms. Metanome and existing implementation are written in Java.
\\

\noindent The proposed algorithms of this thesis are all written in Java as well, enabling us to judge the runtime of the used strategies without needing to account for a language difference. It is known that C or C++ programs are faster and more memory efficient than Java programs. While this is true from a technological stand point, the person writing the code has an arguably higher impact than the language used \cite{prechelt1999technical}. This thesis will discuss strategies and their effect and while the total runtime might be lower in a C/C++ program, the strategies learned are independent of the language of choice. 