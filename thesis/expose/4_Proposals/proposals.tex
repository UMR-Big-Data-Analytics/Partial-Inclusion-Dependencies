% In this document I will collect all open questions which I want to investigate during my thesis
In conclusion, this research will aim to answer several key questions and address specific tasks in order to provide a comprehensive understanding of partial inclusion dependencies (pINDs) in real-world data. These research questions and tasks may evolve throughout the course of the study. \\

\noindent Since pINDs are a mostly unexplored topic, the central research question revolves around finding an efficient pIND discovery algorithm. Modifying the \textit{BINDER} algorithm to enable partial IND discovery, could be a first solution and provide a benchmark for further implementations. Within \textit{BINDER}, we need to adjust the validator to respect partial INDs. Since the source code for the original \textit{BINDER} algorithm is not well structured, we need to perform a rewriting of possible most of the code. \\
% Part on how adjustments could be done

\noindent As discussed in Section \ref{sec:rel_work}, the authors of \textit{SPIDER} also proposed a partial version of their algorithm. We will implement and analyse their proposed algorithm. \\

\noindent Next to \textit{BINDER} and \textit{SPIDER}, we will spent the majority of the research on finding an efficient method for pIND discovery. A first idea is to use a graph structure, which carries meta-data, to invalidate candidates and keep track of existing relations. The thesis will also compare the time it takes to discover pINDs to the time it takes to discover traditional inclusion dependencies (INDs). This will involve investigating the impact of the threshold value on the execution time of pIND discovery algorithms and determining which strategies proposed in past literature can be applied in a partial setting and which strategies will not work. \\

\noindent We will seek to determine the prevalence of pINDs in real-world data and investigate the potential benefits of discovering them. This will involve identifying the datasets in which pINDs are commonly found, as well as establishing a threshold for pIND discovery that is applicable across different dataset domains. We do not expect to find statistically significant findings, since the testing size will most likely be too small for that. The investigation rather aims on getting a first glimpse on the value pINDs could provide.

